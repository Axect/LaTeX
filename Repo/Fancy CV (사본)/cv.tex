%!TEX TS-program = xelatex
\documentclass[]{friggeri-cv}
\addbibresource{bibliography.bib}
\usepackage{kotex}
\begin{document}
\header{Do}{ Jin Kyung}


% In the aside, each new line forces a line break
\begin{aside}
  \section{about}
    Dept. of Electronics
    Engineering
    Ewha Womans Univ.
    1375013
    ~
    jkdo0923@ewhain.net    
  \section{languages}
    Korean
    English
  \section{programming}
    {\color{red} $\varheartsuit$} C++, C, MATLAB, Python, \LaTeX{}
\end{aside}

\section{interests}

\hspace{0.3cm} 정신적, 신체적으로 영향을 주는 뇌신경질환과 직접적으로 연결되는 뇌과학 분야와 의료 영상 처리 및 분석에 관심을 갖고 있습니다. 의료 영상을 처리할 때 보다 높은 효율과 안전성 아래 정확히 원하는 정보를 얻을 수 있는 기술을 개발하는 것에 대해 깊이 배우고 공부하고 싶습니다.

\section{education}

\begin{entrylist}
  \entry
    {2010–2013}
    {정신여자고등학교}
    {Graduate}
    {2013년 2월 졸업.}
  \entry
    {2013 –}
    {B.S. candidate}
    {University}
    {\emph{Dept. of Electronics Engineering, Ewha Univ.}}
\end{entrylist}

\section{experience}

\begin{entrylist}
  \entry
    {2013–2014}
    {이화여자대학교 전자공학과 멘토링 프로그램}
    {Mentee}
    {}
  \entry
    {2014–2015}
    {중앙 가톨릭 동아리 젬마}
    {Leader ship}
    {}
  \entry
    {01–02 2016}
    {Neuroelectronics Engineering Lab, Ewha Womans Univ.}
    {Internship.}
    {Observation of Hindlimb and Tail Movement Induced by Motor Cortex Stimulation(운동피질자극에 의해 유도된 뒷다리와 꼬리 움직임 관찰)}
  \entry
    {07–08 2016}
    { Laboratory of Molecular Neuroimaging Technology (MoNET)}
    {Internship.}
    {- CNN(Convolutional Neural Network)에 대한 학습 \\
     - RNN(Recurrent Neural Network)와 LSTM(Long Short Term Memory) 모델 학습 및 “Schizophrenia Language Modeling-Hallucination”연구 진행 \\
     - Mnist data 변환 처리에 대한 MATLAB code 작성 < CNN(Convolutional Neural Network)를 사용한 Mnist 데이터 변환 > 
    }
\end{entrylist}
\section{Future Plan \& Goals}
\hspace{0.3cm} CT, MRI와 같은 기기를 사용한 검사를 받아본 이후, 이러한 기기들이 어떻게 체내의 정보를 읽어낼 수 있는지에 대한 궁금증을 갖기 시작하면서 고등학생 때부터 뇌과학에 대한 세미나와 특강을 찾아 듣기 시작하였다. 이에 뇌과학 분야에 대해 조금이나마 지식을 쌓게 되었고, 조금 더 깊게 알아보고자 하는 마음으로 전자공학과에 진학하게 되었다. 이후 전공 과목에서 배운 디지털 영상 처리 과정을 기반으로 조직의 성분 특성을 활용하거나 필터링 작업 등을 거쳐 원하는 영상을 더욱 높은 질로 얻을 수 있다는 것을 알게 되었다. 또한, 이번 학기에 바이오전자공학 과목을 수강하면서 기존에 존재하던 MRI, CT, PET 등의 원리를 알 수 있었다. 각 장비들의 조합으로 좋은 결과를 얻을 수 있게 만들어 주는 PET-CT, CT-MRI와 같은 기기들이 등장하여 높은 해상도의 정확한 정보를 얻을 수 있게 되었다는 것을 알게 되었고, 이에 뇌과학과 영상 처리를 사용한 연구를 해보고 싶다.
\pagebreak

\section{Projects}

\begin{entrylist}
  \entry
{2014}
{전자공학기초설계}
{ 바람과 함께 충전되다}
{풍력발전으로 얻어진 에너지를 이용하여 전자기기를 충전할 수 있는 휴대용 충전기를 설계한다. 충전기 역할을 하는 LM2575 소자에 AC 전압이 아닌 DC 전압으로 출력하도록 하는 정전압 회로의 소자인 LM2676과 풍력 발전기 역할을 하는 9개의 fan을 연결하여 풍력 발전기 충전기를 완성한다.}
%------------------------------------------------
\entry
{2015}
{기초회로실험2}
{ 밝기가 점점 흐려지는 LED 점멸기 만들기}
{수업시간에 배운 다이오드와 OPAMP를 활용하여 실생활에 응용할 수 있는 조명등을 제작한다. 전압을 증폭시키는 LM324, BJT인 PNP형 트랜지스터 BC557와 NPN형 트랜지스터 BC547, 470uF 용량의 커패시터, 100kΩ·47kΩ·1kΩ·100Ω 의 저항을 사용하여 회로를 구성하였다.}
%------------------------------------------------
\entry
{2015}
{디지털신호처리및실습}
{ 실루엣 분석을 통한 신체적 특징 감별}
{ MATLAB을 이용하여 사람의 전신 영상을 얻고, 영상을 분석하여 키와 체격 등의 신체 특성을 알아낼 수 있는 프로그램을 제작하였다. }
%------------------------------------------------
\entry
{2015}
{통신공학실험}
{ Counting Different Bits, Check X}
{Checksum과 Exclusive-OR을 사용한 오류 정정 코드를 설계함으로써 디지털 데이터 통신 과정에서 생길 수 있는 오류를 효과적으로 처리할 수 있었다.}
%------------------------------------------------
\entry
{2016}
{디지털영상처리}
{ 밝기 구간 확장 및 밝기 구간에 대한 adaptive 히스토그램 평활화 }
{ HDR 영상에 대해 밝기 구간 별 Histogram Equalization을 실행하여 개선된 화질의 영상을 얻는 Tone mapping 알고리즘을 제시하였다. }
%------------------------------------------------
%------------------------------------------------

\entry
{2016}
{임베디드시스템설계및실습}
{ Ordering in Restaurant using Arduino}
{ 아두이노 간의 XBEE 통신을 이용해 손님과 계산대 간의 직접적인 통신을 가능하게 하고자 하였다. 이를 사용하여 효율성과 정확성이 높은 주문 입출력으로 음식점을 관리할 수 있는 시스템을 만든다.}

\entry
{2016}
{정보시스템프로그래밍및실습}
{ C++을 이용한 성적 관리 프로그램}
{ C++을 사용하여 Object-Oriented Program을 기반으로 계정을 생성하는 클래스와 각 계정의 개인 정보를 담고 있는 클래스를 만들고, 이를 사용하여 수강 내역을 관리할 수 있는 프로그램을 만들었다.}
\end{entrylist}


\end{document}
