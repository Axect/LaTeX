%!TEX TS-program = xelatex
\documentclass[]{friggeri-cv}
\addbibresource{bibliography.bib}
\usepackage{kotex}
\begin{document}

\header{Do}{ Jin Kyung}           % Your name

%----------------------------------------------------------------------------------------
%	SIDEBAR SECTION  -- In the aside, each new line forces a line break
%----------------------------------------------------------------------------------------

\begin{aside}
%
\section{about}
Dept. of Electronics
Engineering
Ewha Womans Univ.
1375013
~
jkdo0923@ewhain.net
%
\section{languages}
Korean
English
%
\section{programming}
C++, C, MATLAB
\LaTeX{}
%
\end{aside}

%----------------------------------------------------------------------------------------
%	SKILLS SECTION
%----------------------------------------------------------------------------------------

\section{interests}
%  \vspace{-0.2cm}

정신적, 신체적으로 영향을 주는 뇌신경질환과 직접적으로 연결되는 뇌과학 분야와 의료 영상 처리 및 분석에 관심을 갖고 있습니다. 의료 영상을 처리할 때 보다 높은 효율과 안전성 아래 정확히 원하는 정보를 얻을 수 있는 기술을 개발하는 것에 대해 깊이 배우고 공부하고 싶습니다.
\section{education}

\begin{entrylist}
\entry
 {2010.03~2013.02}
 {{\normalfont 정신여자고등학교}}
 {Graduate}
 {}
%------------------------------------------------
\entry
 {2013.03~}
 {B.S. {\normalfont candidate in Electronics Engineering}}
 {University}
 {}
%------------------------------------------------
\end{entrylist}


%----------------------------------------------------------------------------------------
%	WORK EXPERIENCE SECTION
%----------------------------------------------------------------------------------------

\section{experience}

\begin{entrylist}
\entry
  {2013.03.01~2014.02.28}
  { 이화여자대학교 전자공학과 멘토링 프로그램}
  {멘티}
  {}
%------------------------------------------------
\entry
  {2014.03.01~2015.03.13}
  { 중앙 가톨릭 동아리 젬마}
  {리더십 활동}
  {집행부 - 총무직\\  총무로서 동아리의 회계 업무를 담당하였다. 동아리 내에서 사용되는 모든 비용을 관리하였으며, 총무 일 뿐만 아니라 집행부로서 다른 집행부원들을 도와 동아리의 행사(총회, 창립제, 축제 부스 운영, 봉사활동)를 모두 관리하였다.}
%------------------------------------------------
\entry
  {2016.01.01~2016.02.29}
  { Neuroelectronics Engineering Lab, Ewha Womans Univ.}
  {Internship.}
  {Observation of Hindlimb and Tail Movement Induced by Motor Cortex Stimulation(운동피질자극에 의해 유도된 뒷다리와 꼬리 움직임 관찰)}}
%------------------------------------------------
\entry
  {2016.07.13~2016.08.31}
  { Laboratory of Molecular Neuroimaging Technology (MoNET)}
  {Internship.}
  {}
\end{entrylist}

%----------------------------------------------------------------------------------------
%	EDUCATION SECTION
%----------------------------------------------------------------------------------------

\section{Future Plan $&$ Goals}
CT, MRI와 같은 기기를 사용한 검사를 받아본 이후, 이러한 기기들이 어떻게 체내의 정보를 읽어낼 수 있는지에 대한 궁금증을 갖기 시작하면서 고등학생 때부터 뇌과학에 대한 세미나와 특강을 찾아 듣기 시작하였다. 이에 뇌과학 분야에 대해 조금이나마 지식을 쌓게 되었고, 조금 더 깊게 알아보고자 하는 마음으로 전자공학과에 진학하게 되었다. 이후 전공 과목에서 배운 디지털 영상 처리 과정을 기반으로 조직의 성분 특성을 활용하거나 필터링 작업 등을 거쳐 원하는 영상을 더욱 높은 질로 얻을 수 있다는 것을 알게 되었다. 또한, 이번 학기에 바이오전자공학 과목을 수강하면서 기존에 존재하던 MRI, CT, PET 등의 원리를 알 수 있었다. 각 장비들의 조합으로 좋은 결과를 얻을 수 있게 만들어 주는 PET-CT, CT-MRI와 같은 기기들이 등장하여 높은 해상도의 정확한 정보를 얻을 수 있게 되었다는 것을 알게 되었고, 이에 뇌과학과 영상 처리를 사용한 연구를 해보고 싶다.
\section{Projects}

\begin{entrylist}
%------------------------------------------------
\entry
{2014}
{전자공학기초설계}
{ 바람과 함께 충전되다}
{풍력발전으로 얻어진 에너지를 이용하여 전자기기를 충전할 수 있는 휴대용 충전기를 설계한다. 충전기 역할을 하는 LM2575 소자에 AC 전압이 아닌 DC 전압으로 출력하도록 하는 정전압 회로의 소자인 LM2676과 풍력 발전기 역할을 하는 9개의 fan을 연결하여 풍력 발전기 충전기를 완성한다.}
%------------------------------------------------
\entry
{2015}
{기초회로실험2}
{ 밝기가 점점 흐려지는 LED 점멸기 만들기}
{수업시간에 배운 다이오드와 OPAMP를 활용하여 실생활에 응용할 수 있는 조명등을 제작한다. 전압을 증폭시키는 LM324, BJT인 PNP형 트랜지스터 BC557와 NPN형 트랜지스터 BC547, 470uF 용량의 커패시터, 100kΩ·47kΩ·1kΩ·100Ω 의 저항을 사용하여 회로를 구성하였다.}
%------------------------------------------------
\entry
{2015}
{디지털신호처리및실습}
{ 실루엣 분석을 통한 신체적 특징 감별}
{ MATLAB을 이용하여 사람의 전신 영상을 얻고, 영상을 분석하여 키와 체격 등의 신체 특성을 알아낼 수 있는 프로그램을 제작하였다. }
%------------------------------------------------
\entry
{2015}
{통신공학실험}
{ Counting Different Bits, Check X}
{Checksum과 Exclusive-OR을 사용한 오류 정정 코드를 설계함으로써 디지털 데이터 통신 과정에서 생길 수 있는 오류를 효과적으로 처리할 수 있었다.}
%------------------------------------------------
\entry
{2016}
{디지털영상처리}
{ 밝기 구간 확장 및 밝기 구간에 대한 adaptive 히스토그램 평활화 }
{ HDR 영상에 대해 밝기 구간 별 Histogram Equalization을 실행하여 개선된 화질의 영상을 얻는 Tone mapping 알고리즘을 제시하였다. }
%------------------------------------------------
\entry
{2016}
{기초회로실험2}
{ 밝기가 점점 흐려지는 LED 점멸기 만들기}
{수업시간에 배운 다이오드와 OPAMP를 활용하여 실생활에 응용할 수 있는 조명등을 제작한다. 전압을 증폭시키는 LM324, BJT인 PNP형 트랜지스터 BC557와 NPN형 트랜지스터 BC547, 470uF 용량의 커패시터, 100kΩ·47kΩ·1kΩ·100Ω 의 저항을 사용하여 회로를 구성하였다.}
%------------------------------------------------
\entry
{2016}
{기초회로실험2}
{ 밝기가 점점 흐려지는 LED 점멸기 만들기}
{수업시간에 배운 다이오드와 OPAMP를 활용하여 실생활에 응용할 수 있는 조명등을 제작한다. 전압을 증폭시키는 LM324, BJT인 PNP형 트랜지스터 BC557와 NPN형 트랜지스터 BC547, 470uF 용량의 커패시터, 100kΩ·47kΩ·1kΩ·100Ω 의 저항을 사용하여 회로를 구성하였다.}
%------------------------------------------------
\entry
{2015}
{기초회로실험2}
{ 밝기가 점점 흐려지는 LED 점멸기 만들기}
{수업시간에 배운 다이오드와 OPAMP를 활용하여 실생활에 응용할 수 있는 조명등을 제작한다. 전압을 증폭시키는 LM324, BJT인 PNP형 트랜지스터 BC557와 NPN형 트랜지스터 BC547, 470uF 용량의 커패시터, 100kΩ·47kΩ·1kΩ·100Ω 의 저항을 사용하여 회로를 구성하였다.}
\end{entrylist}

%----------------------------------------------------------------------------------------
%	OTHER QUALIFICATIONS SECTION
%----------------------------------------------------------------------------------------

% \section{other qualifications}

% \begin{entrylist}
% %------------------------------------------------
% \entry
% {2013}
% {Qualification}
% {Institution}
% {\vspace{-0.3cm}}
% %------------------------------------------------
% \entry
% {2011}
% {Qualification}
% {Institution}
% {\vspace{-0.3cm}}
% %------------------------------------------------
% \end{entrylist}

%----------------------------------------------------------------------------------------
%	AWARDS SECTION
%----------------------------------------------------------------------------------------

\section{awards}

\begin{entrylist}
%------------------------------------------------
\entry
{2014}
{Award name}
{Institution}
{Award description. Award description. Award description. Award description. Award description. Award description. Award description. }
%------------------------------------------------
\end{entrylist}

%----------------------------------------------------------------------------------------
%	INTERESTS SECTION
%----------------------------------------------------------------------------------------

\section{interests}
  \vspace{-0.2cm}

\textbf{professional:} professional interest 1, professional interest 2 and professional interest 3. \textbf{personal:} personal interest 1, personal interest 2, personal interest 3 and personal interest 4.

%----------------------------------------------------------------------------------------

\end{document}