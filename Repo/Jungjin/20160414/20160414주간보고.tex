%%%%%%%%%%%%%%%%%%%%%%%%%%%%%%%%%%%%%%%%%
% Daily Laboratory Book
% LaTeX Template
%
% This template has been downloaded from:
% http://www.latextemplates.com
%
% Original author:
% Frank Kuster (http://www.ctan.org/tex-archive/macros/latex/contrib/labbook/)
%
% Important note:
% This template requires the labbook.cls file to be in the same directory as the
% .tex file. The labbook.cls file provides the necessary structure to create the
% lab book.
%
% The \lipsum[#] commands throughout this template generate dummy text
% to fill the template out. These commands should all be removed when 
% writing lab book content.
%
% HOW TO USE THIS TEMPLATE 
% Each day in the lab consists of three main things:
%
% 1. LABDAY: The first thing to put is the \labday{} command with a date in 
% curly brackets, this will make a new page and put the date in big letters 
% at the top.
%
% 2. EXPERIMENT: Next you need to specify what experiment(s) you are 
% working on with an \experiment{} command with the experiment shorthand 
% in the curly brackets. The experiment shorthand is defined in the 
% 'DEFINITION OF EXPERIMENTS' section below, this means you can 
% say \experiment{pcr} and the actual text written to the PDF will be what 
% you set the 'pcr' experiment to be. If the experiment is a one off, you can 
% just write it in the bracket without creating a shorthand. Note: if you don't 
% want to have an experiment, just leave this out and it won't be printed.
%
% 3. CONTENT: Following the experiment is the content, i.e. what progress 
% you made on the experiment that day.
%
%%%%%%%%%%%%%%%%%%%%%%%%%%%%%%%%%%%%%%%%%

%----------------------------------------------------------------------------------------
%	PACKAGES AND OTHER DOCUMENT CONFIGURATIONS
%----------------------------------------------------------------------------------------

\documentclass[idxtotoc,hyperref,openany]{labbook} % 'openany' here removes the gap page between days, erase it to restore this gap; 'oneside' can also be added to remove the shift that odd pages have to the right for easier reading

\usepackage[ 
  backref=page,
  pdfpagelabels=true,
  plainpages=false,
  colorlinks=true,
  bookmarks=true,
  pdfview=FitB]{hyperref} % Required for the hyperlinks within the PDF
  
\usepackage{booktabs} % Required for the top and bottom rules in the table
\usepackage{float} % Required for specifying the exact location of a figure or table
\usepackage{graphicx} % Required for including images
\usepackage{kotex}
\usepackage{hhline}
\usepackage{multirow}
\usepackage{adjustbox}

\newcommand{\HRule}{\rule{\linewidth}{0.5mm}} % Command to make the lines in the title page
\setlength\parindent{0pt} % Removes all indentation from paragraphs
\makeatletter
\newcommand*{\rom}[1]{\expandafter\@slowromancap\romannumeral #1@}
\makeatother
%----------------------------------------------------------------------------------------
%	DEFINITION OF EXPERIMENTS
%----------------------------------------------------------------------------------------

\newexperiment{cc}{진도 현황}
\newexperiment{at}{수업태도 및 출결상태}
\newexperiment{tb}{교재목록}
\newexperiment{cm}{비고사항}
%\newexperiment{shorthand}{Description of the experiment}

%---------------------------------------------------------------------------------------

\begin{document}

%----------------------------------------------------------------------------------------
%	TITLE PAGE
%----------------------------------------------------------------------------------------

\frontmatter % Use Roman numerals for page numbers
\title{
\begin{center}
\HRule \\[0.4cm]
{\Huge \bfseries 정진학원 주간보고  \\[0.5cm] \Large 고등부 수학}\\[0.4cm] % Degree
\HRule \\[1.5cm]
\end{center}
}
\author{\LARGE 김태근 \\  \\[2cm]} % Your name and email address
\date{2016.04.14} % Beginning date
\maketitle

\tableofcontents

\mainmatter % Use Arabic numerals for page numbers

%----------------------------------------------------------------------------------------
%	LAB BOOK CONTENTS
%----------------------------------------------------------------------------------------

% Blank template to use for new days:

%\labday{Day, Date Month Year}

%\experiment{}

%Text

%-----------------------------------------

%\experiment{}

%Text

%----------------------------------------------------------------------------------------

\labday{2016.04.07 - 2016.04.09}

\experiment{tb}

\begin{table}[h]
\centering
\begin{adjustbox}{width=\textwidth}
\begin{tabular}{c||c|c}
\toprule
\midrule
학년 & \multicolumn{2}{c}{교재 목록} \\
\hhline{=||==}
고1 & 내신 & 메시지 수학1 \\
\hline
고2 & 내신 & 메시지 미적분1\\
\hline
\multirow{3}{*}{고3} & 공통 & 메시지 확률과 통계 \\ \hhline{~--}
					& 문과 & 메시지 미적분1, 리얼 DNA 유형별 확률과통계\\ \hhline{~--}
					& 이과 & 메시지 미적분2, 개념유형 미적분2\\

\hline
\end{tabular}
\end{adjustbox}
\caption{\label{tab:i} 고등부 수학 교재목록 }
\end{table}

\experiment{cc}


\begin{table}[h]
\centering
\begin{adjustbox}{width=.9\textwidth}
\begin{tabular}{c||c|c}
\toprule
\midrule
학년 & \multicolumn{2}{c}{주간 진도 현황} \\
\hhline{=||==}
고1 & 내신 & \rom{2}. 방정식과 부등식 - 3. 이차함수    \\
\hline
\multirow{2}{*}{고2} & 솔터 & 기하와 벡터 \rom{1}. 이차곡선 \\ \hhline{~--}
					& 제일 & 확률과 통계 \rom{1}. 순열과 조합 - 1. 순열 \\
\hline
\multirow{2}{*}{고3} & 문과 & 수학2 - 1. 집합과명제 \\ \hhline{~--}
					& 이과 & \rom{2}. 삼각함수 - 2. 삼각함수의 그래프  \\

\hline
\end{tabular}
\end{adjustbox}
\caption{\label{tab:ii} 고등부 수학 진도 현황 }
\end{table}

%-----------------------------------------

\experiment{at} % Multiple experiments can be included in a single day, this allows you to segment what was done each day into separate categories



\begin{table}[H]
\centering
\begin{adjustbox}{width=\textwidth}
\begin{tabular}{c|c||c|c|c|c|c}
\toprule
\midrule
학년 & 이름 & 결석 & 숙제불이행 & 교재미지참 & 수업태도 & 종합태도점수 \\
\hhline{=|=||=|=|=|=|=}
\multirow{5}{*}{고1}& 김동현 & 0 & 1 & 0 & 5 & 95  \\ \hhline{~------}
					& 김정욱 & 0 & 1 & 0 & 2 & 59   \\ \hhline{~------}
					& 김주희 & 0 & 1 & 0 & 5 & 95   \\ \hhline{~------}
					& 손예진 & - & - & - & - & -  \\ \hhline{~------}
					& 윤선영 & 1 & 1 & 0 & 4 & 78   \\
\hline
\multirow{2}{*}{고2}& 김미연 & 0 & 1 & 0 & 5 & 95   \\ \hhline{~------}
					& 안상호 & 0 & 1 & 0 & 5 & 95	\\ 
\hline
\multirow{2}{*}{고3}& 김현우 & 1 & 0 & 0 & 5 & 95  \\ \hhline{~------}
					& 장한수 & 1 & 1 & 0 & 5 & 90	\\
\hline
\end{tabular}
\end{adjustbox}
\caption{\label{tab:iii} 고등부 태도 현황 }
\end{table}

%-----------------------------------------

\experiment{cm}
\begin{itemize}
	\item 고1,고2,고3 모두 수업태도가 양호하였음.
	\item 고1에서는 동현이와 선영이, 그리고 주희의 차이가 심했는데 시험기간이라 좁혀지는 중.
	\item 고2에서는 미연이가 확률에 강한 모습을 보이며 상호 역시 기하와 벡터는 어느 정도 함.
	\item 고3에서는 현우가 시험범위인 고1 수학 2에 약한 모습을 보여 보완이 필요함.
	\item 한수는 미적분2는 어느 정도 따라오나. 확률과 통계가 문제가 될 수 있음.
	\item 
\end{itemize}

\experiment{일정공지}


제 중간고사 일정은 현재 나온 결과, 4/19(화) 낮 11시 - 13시, 4/21(목) 저녁 7시 - 9시, 4/23(토) 오후 1시 - 3시입니다. 여기서 토요일 시험은 존재하지 않을 수 있으나, 목요일 시험은 확정이므로 다음 주의 일정 조정이 필요할 것 같습니다. 토요일 시험이 없다면 목요일 시간만 화요일로 옮기면 되겠지만, 토요일 시험이 존재한다면 목요일 시간을 화요일로 옮기는 것 뿐만 아니라, 토요일 강의 시간 역시 토요일 오후 5시 정도로 옮겨야 할 것 같습니다.


%-----------------------------------------


%----------------------------------------------------------------------------------------
%	FORMULAE AND MEDIA RECIPES
%----------------------------------------------------------------------------------------

------------------------------------------
%	MEDIA RECIPES
%----------------------------------------------------------------------------------------


%-----------------------------------------

%\textbf{Media 2}\\ \\

%Description

%----------------------------------------------------------------------------------------
%	FORMULAE
%----------------------------------------------------------------------------------------


%-----------------------------------------

%\textbf{Formula X - Description}\\ \\

%Formula

%----------------------------------------------------------------------------------------

\end{document}