%%%%%%%%%%%%%%%%%%%%%%%%%%%%%%%%%%%%%%%%%
% Daily Laboratory Book
% LaTeX Template
%
% This template has been downloaded from:
% http://www.latextemplates.com
%
% Original author:
% Frank Kuster (http://www.ctan.org/tex-archive/macros/latex/contrib/labbook/)
%
% Important note:
% This template requires the labbook.cls file to be in the same directory as the
% .tex file. The labbook.cls file provides the necessary structure to create the
% lab book.
%
% The \lipsum[#] commands throughout this template generate dummy text
% to fill the template out. These commands should all be removed when 
% writing lab book content.
%
% HOW TO USE THIS TEMPLATE 
% Each day in the lab consists of three main things:
%
% 1. LABDAY: The first thing to put is the \labday{} command with a date in 
% curly brackets, this will make a new page and put the date in big letters 
% at the top.
%
% 2. EXPERIMENT: Next you need to specify what experiment(s) you are 
% working on with an \experiment{} command with the experiment shorthand 
% in the curly brackets. The experiment shorthand is defined in the 
% 'DEFINITION OF EXPERIMENTS' section below, this means you can 
% say \experiment{pcr} and the actual text written to the PDF will be what 
% you set the 'pcr' experiment to be. If the experiment is a one off, you can 
% just write it in the bracket without creating a shorthand. Note: if you don't 
% want to have an experiment, just leave this out and it won't be printed.
%
% 3. CONTENT: Following the experiment is the content, i.e. what progress 
% you made on the experiment that day.
%
%%%%%%%%%%%%%%%%%%%%%%%%%%%%%%%%%%%%%%%%%

%----------------------------------------------------------------------------------------
%	PACKAGES AND OTHER DOCUMENT CONFIGURATIONS
%----------------------------------------------------------------------------------------

\documentclass[idxtotoc,hyperref,openany]{labbook} % 'openany' here removes the gap page between days, erase it to restore this gap; 'oneside' can also be added to remove the shift that odd pages have to the right for easier reading

\usepackage[ 
  backref=page,
  pdfpagelabels=true,
  plainpages=false,
  colorlinks=true,
  bookmarks=true,
  pdfview=FitB]{hyperref} % Required for the hyperlinks within the PDF
  
\usepackage{booktabs} % Required for the top and bottom rules in the table
\usepackage{float} % Required for specifying the exact location of a figure or table
\usepackage{graphicx} % Required for including images
\usepackage{kotex}
\usepackage{hhline}
\usepackage{multirow}
\usepackage{adjustbox}

\newcommand{\HRule}{\rule{\linewidth}{0.5mm}} % Command to make the lines in the title page
\setlength\parindent{0pt} % Removes all indentation from paragraphs
\makeatletter
\newcommand*{\rom}[1]{\expandafter\@slowromancap\romannumeral #1@}
\makeatother
%----------------------------------------------------------------------------------------
%	DEFINITION OF EXPERIMENTS
%----------------------------------------------------------------------------------------

\newexperiment{cc}{진도 현황}
\newexperiment{at}{수업태도 및 출결상태}
\newexperiment{tb}{교재목록}
\newexperiment{cm}{비고사항}
%\newexperiment{shorthand}{Description of the experiment}

%---------------------------------------------------------------------------------------

\begin{document}

%----------------------------------------------------------------------------------------
%	TITLE PAGE
%----------------------------------------------------------------------------------------

\frontmatter % Use Roman numerals for page numbers
\title{
\begin{center}
\HRule \\[0.4cm]
{\Huge \bfseries 정진학원 주간보고  \\[0.5cm] \Large 고등부 수학}\\[0.4cm] % Degree
\HRule \\[1.5cm]
\end{center}
}
\author{\LARGE 김태근 \\  \\[2cm]} % Your name and email address
\date{2016.05.16} % Beginning date
\maketitle

\tableofcontents

\mainmatter % Use Arabic numerals for page numbers

%----------------------------------------------------------------------------------------
%	LAB BOOK CONTENTS
%----------------------------------------------------------------------------------------

% Blank template to use for new days:

%\labday{Day, Date Month Year}

%\experiment{}

%Text

%-----------------------------------------

%\experiment{}

%Text

%----------------------------------------------------------------------------------------

\labday{2016.05.11 - 2016.05.14}

\experiment{tb}

\begin{table}[h]
\centering
\begin{adjustbox}{width=\textwidth}
\begin{tabular}{c||c|c}
\toprule
\midrule
학년 & \multicolumn{2}{c}{교재 목록} \\
\hhline{=||==}
고1 & 내신 & 메시지 수학1 \\
\hline
고2 & 내신 & 메시지 미적분1\\
\hline
\multirow{3}{*}{고3} & 공통 & 메시지 확률과 통계 \\ \hhline{~--}
					& 문과 & 메시지 미적분1, 리얼 DNA 유형별 확률과통계\\ \hhline{~--}
					& 이과 & 메시지 미적분2, 개념유형 미적분2\\

\hline
\end{tabular}
\end{adjustbox}
\caption{\label{tab:i} 고등부 수학 교재목록 }
\end{table}

\experiment{cc}


\begin{table}[h]
\centering
\begin{adjustbox}{width=.9\textwidth}
\begin{tabular}{c||c|c}
\toprule
\midrule
학년 & \multicolumn{2}{c}{주간 진도 현황} \\
\hhline{=||==}
고1 & 내신 & \rom{2}. 방정식과 부등식 - 5. 여러가지 부등식   \\
\hline
고2 & 내신 & \rom{3}. 다항함수의 미분법 - 1. 미분계수와 도함수 \\
\hline
\multirow{3}{*}{고3} & 공통 & 모의고사 대비 (함수의 극한과 연속) \\ \hhline{~--}
					& 문과 &  \\ \hhline{~--}
					& 이과 & \rom{2}. 삼각함수 - 3. 삼각함수의 미분법  \\

\hline
\end{tabular}
\end{adjustbox}
\caption{\label{tab:ii} 고등부 수학 진도 현황 }
\end{table}

%-----------------------------------------

\experiment{at} % Multiple experiments can be included in a single day, this allows you to segment what was done each day into separate categories



\begin{table}[H]
\centering
\begin{adjustbox}{width=\textwidth}
\begin{tabular}{c|c||c|c|c|c|c}
\toprule
\midrule
학년 & 이름 & 결석 & 숙제불이행 & 교재미지참 & 수업태도 & 종합태도점수 \\
\hhline{=|=||=|=|=|=|=}
\multirow{5}{*}{고1}& 김동현 & 0 & 0 & 0 & 3 & 76  \\ \hhline{~------}
					& 김정욱 & - & - & - & - & -   \\ \hhline{~------}
					& 김주희 & 0.5 & 0 & 0 & 5 & 97.5   \\ \hhline{~------}
					& 손예진 & 0 & 0 & 0 & 5 & 100  \\ \hhline{~------}
					& 윤선영 & 0 & 0 & 0 & 4 & 88   \\
\hline
\multirow{2}{*}{고2}& 김미연 & 0 & 0 & 0 & 5 & 100   \\ \hhline{~------}
					& 안상호 & 1 & 0 & 0 & 4 & 83	\\ 
\hline
\multirow{2}{*}{고3}& 김현우 & 0 & 0 & 0 & 5 & 100  \\ \hhline{~------}
					& 장한수 & 0 & 0 & 0 & 5 & 100	\\
\hline
\end{tabular}
\end{adjustbox}
\caption{\label{tab:iii} 고등부 태도 현황 }
\end{table}

%-----------------------------------------

\experiment{cm}
\begin{itemize}
	\item 이번 주 수업태도는 모든 학년이 양호하였음.
	\item 선영이의 수업태도가 상당히 좋아졌음. 그러나 동현이의 수업태도는 아직 별로 좋지 않음.
	\item 고1 모의고사(2015년 6월) 시행 결과 주희는 63점으로 4등급, 선영이는 49 점으로 5등급, 동현이는 28점으로 7등급을 기록함.
	\item 고2의 경우 미연이의 수업태도는 우수하였고 상호는 결석으로 진도를 나가지 못함.
	\item 고3의 경우 미적분1 테스트를 진행하였는데 현우는 꽤 기억하고 있었으나 한수는 상당부분을 잊어버려서 보완이 필요함.
\end{itemize}

\experiment{방학일정}
두 가지 방학일정이 나왔습니다. 하나는 방학시작하고 1주일 동안 학회가 잡혀있는 것이고 하나는 계절학기입니다.
\begin{itemize}
 \item 6월 26일 (일) $-$ 7월 2일 (토) : APCTP (아시아태평양이론물리센터) 에서 주최하는 수치상대론학회가 있습니다. 대전 KAIST에서 1주일 동안 진행되기에 불가피하게 학원을 빠질 수 밖에 없습니다.
 \item 7월 4일 (월) $-$ 7월 15일 (금) : 계절학기 관계로 매일 1시 $-$ 5시 까지 수업이 있습니다. 그러나 아직 학생들은 방학이 아니니 상관없을 듯 합니다.
\end{itemize}


%-----------------------------------------


%----------------------------------------------------------------------------------------
%	FORMULAE AND MEDIA RECIPES
%----------------------------------------------------------------------------------------

------------------------------------------
%	MEDIA RECIPES
%----------------------------------------------------------------------------------------


%-----------------------------------------

%\textbf{Media 2}\\ \\

%Description

%----------------------------------------------------------------------------------------
%	FORMULAE
%----------------------------------------------------------------------------------------


%-----------------------------------------

%\textbf{Formula X - Description}\\ \\

%Formula

%----------------------------------------------------------------------------------------

\end{document}