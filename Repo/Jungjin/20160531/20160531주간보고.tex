%%%%%%%%%%%%%%%%%%%%%%%%%%%%%%%%%%%%%%%%%
% Daily Laboratory Book
% LaTeX Template
%
% This template has been downloaded from:
% http://www.latextemplates.com
%
% Original author:
% Frank Kuster (http://www.ctan.org/tex-archive/macros/latex/contrib/labbook/)
%
% Important note:
% This template requires the labbook.cls file to be in the same directory as the
% .tex file. The labbook.cls file provides the necessary structure to create the
% lab book.
%
% The \lipsum[#] commands throughout this template generate dummy text
% to fill the template out. These commands should all be removed when 
% writing lab book content.
%
% HOW TO USE THIS TEMPLATE 
% Each day in the lab consists of three main things:
%
% 1. LABDAY: The first thing to put is the \labday{} command with a date in 
% curly brackets, this will make a new page and put the date in big letters 
% at the top.
%
% 2. EXPERIMENT: Next you need to specify what experiment(s) you are 
% working on with an \experiment{} command with the experiment shorthand 
% in the curly brackets. The experiment shorthand is defined in the 
% 'DEFINITION OF EXPERIMENTS' section below, this means you can 
% say \experiment{pcr} and the actual text written to the PDF will be what 
% you set the 'pcr' experiment to be. If the experiment is a one off, you can 
% just write it in the bracket without creating a shorthand. Note: if you don't 
% want to have an experiment, just leave this out and it won't be printed.
%
% 3. CONTENT: Following the experiment is the content, i.e. what progress 
% you made on the experiment that day.
%
%%%%%%%%%%%%%%%%%%%%%%%%%%%%%%%%%%%%%%%%%

%----------------------------------------------------------------------------------------
%	PACKAGES AND OTHER DOCUMENT CONFIGURATIONS
%----------------------------------------------------------------------------------------

\documentclass[idxtotoc,hyperref,openany]{labbook} % 'openany' here removes the gap page between days, erase it to restore this gap; 'oneside' can also be added to remove the shift that odd pages have to the right for easier reading

\usepackage[ 
  backref=page,
  pdfpagelabels=true,
  plainpages=false,
  colorlinks=true,
  bookmarks=true,
  pdfview=FitB]{hyperref} % Required for the hyperlinks within the PDF
  
\usepackage{booktabs} % Required for the top and bottom rules in the table
\usepackage{float} % Required for specifying the exact location of a figure or table
\usepackage{graphicx} % Required for including images
\usepackage{kotex}
\usepackage{hhline}
\usepackage{multirow}
\usepackage{adjustbox}

\newcommand{\HRule}{\rule{\linewidth}{0.5mm}} % Command to make the lines in the title page
\setlength\parindent{0pt} % Removes all indentation from paragraphs
\makeatletter
\newcommand*{\rom}[1]{\expandafter\@slowromancap\romannumeral #1@}
\makeatother
%----------------------------------------------------------------------------------------
%	DEFINITION OF EXPERIMENTS
%----------------------------------------------------------------------------------------

\newexperiment{cc}{진도 현황}
\newexperiment{at}{수업태도 및 출결상태}
\newexperiment{tb}{교재목록}
\newexperiment{cm}{비고사항}
%\newexperiment{shorthand}{Description of the experiment}

%---------------------------------------------------------------------------------------

\begin{document}

%----------------------------------------------------------------------------------------
%	TITLE PAGE
%----------------------------------------------------------------------------------------

\frontmatter % Use Roman numerals for page numbers
\title{
\begin{center}
\HRule \\[0.4cm]
{\Huge \bfseries 정진학원 주간보고  \\[0.5cm] \Large 고등부 수학}\\[0.4cm] % Degree
\HRule \\[1.5cm]
\end{center}
}
\author{\LARGE 김태근 \\  \\[2cm]} % Your name and email address
\date{\today} % Beginning date
\maketitle

\tableofcontents

\mainmatter % Use Arabic numerals for page numbers

%----------------------------------------------------------------------------------------
%	LAB BOOK CONTENTS
%----------------------------------------------------------------------------------------

% Blank template to use for new days:

%\labday{Day, Date Month Year}

%\experiment{}

%Text

%-----------------------------------------

%\experiment{}

%Text

%----------------------------------------------------------------------------------------

\labday{2016.05.26 $-$ 2016.05.28}

\experiment{tb}

\begin{table}[h]
\centering
\begin{adjustbox}{width=\textwidth}
\begin{tabular}{c||c|c}
\toprule
\midrule
학년 & \multicolumn{2}{c}{교재 목록} \\
\hhline{=||==}
고1 & 내신 & 메시지 수학1 \\
\hline
고2 & 내신 & 메시지 미적분1\\
\hline
\multirow{3}{*}{고3} & 공통 & 메시지 확률과 통계 \\ \hhline{~--}
					& 문과 & 메시지 미적분1, 리얼 DNA 유형별 확률과통계\\ \hhline{~--}
					& 이과 & 메시지 미적분2, 개념유형 미적분2\\

\hline
\end{tabular}
\end{adjustbox}
\caption{\label{tab:i} 고등부 수학 교재목록 }
\end{table}

\experiment{cc}


\begin{table}[h]
\centering
\begin{adjustbox}{width=.9\textwidth}
\begin{tabular}{c||c|c}
\toprule
\midrule
학년 & \multicolumn{2}{c}{주간 진도 현황} \\
\hhline{=||==}
고1 & 내신 & \rom{2}. 방정식과 부등식 - 5. 여러가지 부등식   \\
\hline
고2 & 내신 & \rom{3}. 다항함수의 미분법 - 1. 미분계수와 도함수 \\
\hline
\multirow{3}{*}{고3} & 공통 & 모의고사 대비  \\ \hhline{~--}
					& 문과 &  \\ \hhline{~--}
					& 이과 & \rom{2}. 삼각함수 - 3. 삼각함수의 미분법  \\

\hline
\end{tabular}
\end{adjustbox}
\caption{\label{tab:ii} 고등부 수학 진도 현황 }
\end{table}

%-----------------------------------------

\experiment{at} % Multiple experiments can be included in a single day, this allows you to segment what was done each day into separate categories



\begin{table}[H]
\centering
\begin{adjustbox}{width=\textwidth}
\begin{tabular}{c|c||c|c|c|c|c}
\toprule
\midrule
학년 & 이름 & 결석 & 숙제불이행 & 교재미지참 & 수업태도 & 종합태도점수 \\
\hhline{=|=||=|=|=|=|=}
\multirow{3}{*}{고1}			& 김주희 & 1 & 0 & 0 & 5 & 95   \\ \hhline{~------}
					& 손예진 & 0 & 1 & 0 & 5 & 95  \\ \hhline{~------}
					& 윤선영 & 0 & 0 & 0 & 5 & 100   \\
\hline
\multirow{1}{*}{고2}			& 안상호 & 1 & 0 & 0 & 4 & 83	\\ 
\hline
\multirow{2}{*}{고3}			& 김현우 & 0 & 1 & 0 & 5 & 95  \\ \hhline{~------}
					& 장한수 & 0 & 1 & 0 & 5 & 95	\\
\hline
\end{tabular}
\end{adjustbox}
\caption{\label{tab:iii} 고등부 태도 현황 }
\end{table}

%-----------------------------------------

\experiment{cm}
\begin{itemize}
	\item 이번 주는 다들 어느 정도 산만해서 수업집중도가 떨어졌음.
	\item 상대적으로 결석이 빈번했으나 고3의 경우는 다들 수업에 열심히 임함.
	\item 고2의 경우 상호가 결석 하기도 하였고 아파서 조퇴도 하였으므로 진도를 나가지 못함.
	\item 고3의 경우 모의고사 대비 수업을 진행하였는데 결과가 꽤 좋지 못하여 앞으로 적극적인 보강이 필요함.
\end{itemize}

\experiment{교재 구입 예정 목록}


\hspace{1cm}이제 목요일이면 6월 모의고사가 끝나게 되고 고3은 입시대비를 해야합니다. 그러나 중상위 이상의 대학들에는 대부분 최저학력이 존재하며 
한수나 현우 역시 수시 원서의 반 정도는 최저가 있는 학교를 쓸지 모릅니다. 따라서 이제 기본 실력이 어느 정도 닦였으니 EBS를 풀어야 할 것 같습니다.
다만 한수의 경우 솔터고에서 수학 가형을 대비해주지 않기에 수학 나형을 보아야 할지도 모르니 아직 교재는 미정입니다. 이는 이번 주 내로 결정하면 될 것 같습니다.

\begin{table}[H]
\centering
\begin{adjustbox}{width=\textwidth}
 \begin{tabular}{c||c|c}
 \toprule
 \midrule
 학년 & \multicolumn{2}{c}{교재목록} \\
 \hhline{=||==}
 \multirow{2}{*}{고3} & 문과 & 수능특강 수학2 \& 미적분1, 확률과통계, EBS N제 수학(나)형  \\ \hhline{~--}
		      & 이과 & 미정 (6월 모의고사 후 결정) \\
 \hline
  
 \end{tabular}

\end{adjustbox}
\caption{\label{tab:iv} 고3 교재구입목록}
\end{table}


\experiment{시험기간 일정}
아직 시험일정이 하나밖에 안나와서 추후에 또 말씀드리겠습니다.
\begin{itemize}
 \item 6월 8일 (수) 저녁 7시 시험 : 따라서 고3은 이 날 수업을 진행하지 못할 듯 합니다.
\end{itemize}


%-----------------------------------------


%----------------------------------------------------------------------------------------
%	FORMULAE AND MEDIA RECIPES
%----------------------------------------------------------------------------------------

------------------------------------------
%	MEDIA RECIPES
%----------------------------------------------------------------------------------------


%-----------------------------------------

%\textbf{Media 2}\\ \\

%Description

%----------------------------------------------------------------------------------------
%	FORMULAE
%----------------------------------------------------------------------------------------


%-----------------------------------------

%\textbf{Formula X - Description}\\ \\

%Formula

%----------------------------------------------------------------------------------------

\end{document}