%%%%%%%%%%%%%%%%%%%%%%%%%%%%%%%%%%%%%%%%%
% Daily Laboratory Book
% LaTeX Template
%
% This template has been downloaded from:
% http://www.latextemplates.com
%
% Original author:
% Frank Kuster (http://www.ctan.org/tex-archive/macros/latex/contrib/labbook/)
%
% Important note:
% This template requires the labbook.cls file to be in the same directory as the
% .tex file. The labbook.cls file provides the necessary structure to create the
% lab book.
%
% The \lipsum[#] commands throughout this template generate dummy text
% to fill the template out. These commands should all be removed when 
% writing lab book content.
%
% HOW TO USE THIS TEMPLATE 
% Each day in the lab consists of three main things:
%
% 1. LABDAY: The first thing to put is the \labday{} command with a date in 
% curly brackets, this will make a new page and put the date in big letters 
% at the top.
%
% 2. EXPERIMENT: Next you need to specify what experiment(s) you are 
% working on with an \experiment{} command with the experiment shorthand 
% in the curly brackets. The experiment shorthand is defined in the 
% 'DEFINITION OF EXPERIMENTS' section below, this means you can 
% say \experiment{pcr} and the actual text written to the PDF will be what 
% you set the 'pcr' experiment to be. If the experiment is a one off, you can 
% just write it in the bracket without creating a shorthand. Note: if you don't 
% want to have an experiment, just leave this out and it won't be printed.
%
% 3. CONTENT: Following the experiment is the content, i.e. what progress 
% you made on the experiment that day.
%
%%%%%%%%%%%%%%%%%%%%%%%%%%%%%%%%%%%%%%%%%

%----------------------------------------------------------------------------------------
%	PACKAGES AND OTHER DOCUMENT CONFIGURATIONS
%----------------------------------------------------------------------------------------

\documentclass[idxtotoc,hyperref,openany]{labbook} % 'openany' here removes the gap page between days, erase it to restore this gap; 'oneside' can also be added to remove the shift that odd pages have to the right for easier reading

\usepackage[ 
  backref=page,
  pdfpagelabels=true,
  plainpages=false,
  colorlinks=true,
  bookmarks=true,
  pdfview=FitB]{hyperref} % Required for the hyperlinks within the PDF
  
\usepackage{booktabs} % Required for the top and bottom rules in the table
\usepackage{float} % Required for specifying the exact location of a figure or table
\usepackage{graphicx} % Required for including images
\usepackage{kotex}
\usepackage{hhline}
\usepackage{multirow}
\usepackage{adjustbox}

\newcommand{\HRule}{\rule{\linewidth}{0.5mm}} % Command to make the lines in the title page
\setlength\parindent{0pt} % Removes all indentation from paragraphs
\makeatletter
\newcommand*{\rom}[1]{\expandafter\@slowromancap\romannumeral #1@}
\makeatother
%----------------------------------------------------------------------------------------
%	DEFINITION OF EXPERIMENTS
%----------------------------------------------------------------------------------------

\newexperiment{cc}{진도 현황}
\newexperiment{at}{수업태도 및 출결상태}
\newexperiment{tb}{교재목록}
\newexperiment{cm}{비고사항}
\newexperiment{pl}{2학기 일정}
%\newexperiment{shorthand}{Description of the experiment}

%---------------------------------------------------------------------------------------

\begin{document}

%----------------------------------------------------------------------------------------
%	TITLE PAGE
%----------------------------------------------------------------------------------------

\frontmatter % Use Roman numerals for page numbers
\title{
\begin{center}
\HRule \\[0.4cm]
{\Huge \bfseries 정진학원 주간보고  \\[0.5cm] \Large 고등부 수학}\\[0.4cm] % Degree
\HRule \\[1.5cm]
\end{center}
}
\author{\LARGE 김태근 \\  \\[2cm]} % Your name and email address
\date{\today} % Beginning date
\maketitle

\tableofcontents

\mainmatter % Use Arabic numerals for page numbers

%----------------------------------------------------------------------------------------
%	LAB BOOK CONTENTS
%----------------------------------------------------------------------------------------

% Blank template to use for new days:

%\labday{Day, Date Month Year}

%\experiment{}

%Text

%-----------------------------------------

%\experiment{}

%Text

%----------------------------------------------------------------------------------------

\labday{2016.07.28 $-$ 2016.08.09}

\experiment{tb}

\begin{table}[h]
\centering
\begin{adjustbox}{width=.9\textwidth}
\begin{tabular}{c||c|c}
\toprule
\midrule
학년 & \multicolumn{2}{c}{교재 목록} \\
\hhline{=||==}
고1 & 선행 & 개념유형 수학2 \& 메시지 수학2 \\
\hline
고2 & 선행 & 개념유형 미적분1 \& 메시지 미적분1\\
\hline
\multirow{1}{*}{고3} & 공통 & 수능특강 수학2 \& 미적분1, 확률과통계 \\

\hline
\end{tabular}
\end{adjustbox}
\caption{\label{tab:i} 고등부 수학 교재목록 }
\end{table}

\experiment{cc}


\begin{table}[h]
\centering
\begin{adjustbox}{width=\textwidth}
\begin{tabular}{c||c|c}
\toprule
\midrule
학년 & \multicolumn{2}{c}{주간 진도 현황} \\
\hhline{=||==}
고1 & 선행 & 수학2 - 함수\\
\hline
고2 & 선행 & 미적분1 - 도함수의 활용 - 5. 그래프 $-$ 7. 속도와 가속도\\
\hline
\multirow{2}{*}{고3} & 수능 & 수능특강 수학2 - 수학적 귀납법, 지수, 로그, 수열의 극한, 급수 \\ \hhline{~--}
                     & 적성 & 가천대학교 2016년 적성고사 실시 \\

\hline
\end{tabular}
\end{adjustbox}
\caption{\label{tab:ii} 고등부 수학 진도 현황 }
\end{table}

%-----------------------------------------

\experiment{at} % Multiple experiments can be included in a single day, this allows you to segment what was done each day into separate categories



\begin{table}[H]
\centering
\begin{adjustbox}{width=\textwidth}
\begin{tabular}{c|c||c|c|c|c|c}
\toprule
\midrule
학년 & 이름 & 결석 & 숙제불이행 & 교재미지참 & 수업태도 & 종합태도점수 \\
\hhline{=|=||=|=|=|=|=}
\multirow{4}{*}{고1}			& 김주희 & 0 & 2 & 0 & 5 & 90   \\ \hhline{~------}
					& 손예진 & - & - & - & - & -  \\ \hhline{~------}
					& 이승현 & 0 & 0 & 0 & 5 & 100  \\ \hhline{~------}
					& 양진혁 & 0 & 0 & 0 & 5 & 100 \\
\hline
\multirow{1}{*}{고2}			& 안상호 & 0 & 1 & 0 & 5 & 95	\\ 
\hline
\multirow{2}{*}{고3}			& 김현우 & 0 & 1 & 0 & 5 & 100  \\ \hhline{~------}
					& 장한수 & 0 & 1 & 0 & 4 & 83	\\
\hline
\end{tabular}
\end{adjustbox}
\caption{\label{tab:iii} 고등부 태도 현황 }
\end{table}

%-----------------------------------------

\experiment{cm}
\begin{itemize}
	\item 고1은 진혁이가 들어온 후에는 수업태도가 모두 양호함. 승현이가 두각을 보이며 모두들 잘 따라오고 있는 중.
	\item 고2는 수업때 배운 내용으로 수업을 실시했는데 점수는 대략 70점 정도로 이해는 했으나 아직 활용이 제대로 되지 않는 중. 앞으로 문제를 계속
	푼다면 2학기도 기대해볼 수 있을 것 같음.
	\item 고3은 진도는 무난히 나가고 있고, 화요일에 가천대학교 2016 적성고사 기출문제 (국어 20, 수학 20, 영어 10/ 60분) 로 시험을 보았는데, 현우는
	34/50, 한수는 25/50 으로 다들 불합격선 점수를 받음. (합격선은 40이상) 특히, 요즘 공부를 하지 않은 수학 뒷부분 (적분, 통계) 과 영어 어법에서 거의 대부분 틀림.
	따라서 이에 대해 보완이 시급해보임.
\end{itemize}


%-----------------------------------------

\experiment{pl}
\begin{itemize}
 \item 2학기 시간표가 확정되었습니다. 월요일은 연구실 랩미팅이 예정되어 있고, 화요일은 수업이 8시 종료, 수요일은 수업이 7시에 끝나고 목,금은 5시 전에
 수업이 끝납니다. 따라서 월,화는 일단 현실적으로 불가능하고 저번 학기처럼 수요일 저녁 8시 반쯤 고3을 잡고, 나머지는 목,토 혹은
 금,토 로 수업을 잡아야 할 것 같습니다.
\end{itemize}


%----------------------------------------------------------------------------------------
%	FORMULAE AND MEDIA RECIPES
%----------------------------------------------------------------------------------------

------------------------------------------
%	MEDIA RECIPES
%----------------------------------------------------------------------------------------


%-----------------------------------------

%\textbf{Media 2}\\ \\

%Description

%----------------------------------------------------------------------------------------
%	FORMULAE
%----------------------------------------------------------------------------------------


%-----------------------------------------

%\textbf{Formula X - Description}\\ \\

%Formula

%----------------------------------------------------------------------------------------

\end{document}