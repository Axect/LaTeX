%%%%%%%%%%%%%%%%%%%%%%%%%%%%%%%%%%%%%%%%%
% Daily Laboratory Book
% LaTeX Template
%
% This template has been downloaded from:
% http://www.latextemplates.com
%
% Original author:
% Frank Kuster (http://www.ctan.org/tex-archive/macros/latex/contrib/labbook/)
%
% Important note:
% This template requires the labbook.cls file to be in the same directory as the
% .tex file. The labbook.cls file provides the necessary structure to create the
% lab book.
%
% The \lipsum[#] commands throughout this template generate dummy text
% to fill the template out. These commands should all be removed when 
% writing lab book content.
%
% HOW TO USE THIS TEMPLATE 
% Each day in the lab consists of three main things:
%
% 1. LABDAY: The first thing to put is the \labday{} command with a date in 
% curly brackets, this will make a new page and put the date in big letters 
% at the top.
%
% 2. EXPERIMENT: Next you need to specify what experiment(s) you are 
% working on with an \experiment{} command with the experiment shorthand 
% in the curly brackets. The experiment shorthand is defined in the 
% 'DEFINITION OF EXPERIMENTS' section below, this means you can 
% say \experiment{pcr} and the actual text written to the PDF will be what 
% you set the 'pcr' experiment to be. If the experiment is a one off, you can 
% just write it in the bracket without creating a shorthand. Note: if you don't 
% want to have an experiment, just leave this out and it won't be printed.
%
% 3. CONTENT: Following the experiment is the content, i.e. what progress 
% you made on the experiment that day.
%
%%%%%%%%%%%%%%%%%%%%%%%%%%%%%%%%%%%%%%%%%

%----------------------------------------------------------------------------------------
%	PACKAGES AND OTHER DOCUMENT CONFIGURATIONS
%----------------------------------------------------------------------------------------

\documentclass[idxtotoc,hyperref,openany]{labbook} % 'openany' here removes the gap page between days, erase it to restore this gap; 'oneside' can also be added to remove the shift that odd pages have to the right for easier reading

\usepackage[ 
  backref=page,
  pdfpagelabels=true,
  plainpages=false,
  colorlinks=true,
  bookmarks=true,
  pdfview=FitB]{hyperref} % Required for the hyperlinks within the PDF
  
\usepackage{booktabs} % Required for the top and bottom rules in the table
\usepackage{float} % Required for specifying the exact location of a figure or table
\usepackage{graphicx} % Required for including images
\usepackage{kotex}
\usepackage{hhline}
\usepackage{multirow}
\usepackage{adjustbox}

\newcommand{\HRule}{\rule{\linewidth}{0.5mm}} % Command to make the lines in the title page
\setlength\parindent{0pt} % Removes all indentation from paragraphs
\makeatletter
\newcommand*{\rom}[1]{\expandafter\@slowromancap\romannumeral #1@}
\makeatother
%----------------------------------------------------------------------------------------
%	DEFINITION OF EXPERIMENTS
%----------------------------------------------------------------------------------------

\newexperiment{cc}{진도 현황}
\newexperiment{at}{수업태도 및 출결상태}
\newexperiment{tb}{교재목록}
\newexperiment{cm}{비고사항}
\newexperiment{bo}{교재구입}
%\newexperiment{shorthand}{Description of the experiment}

%---------------------------------------------------------------------------------------

\begin{document}

%----------------------------------------------------------------------------------------
%	TITLE PAGE
%----------------------------------------------------------------------------------------

\frontmatter % Use Roman numerals for page numbers
\title{
\begin{center}
\HRule \\[0.4cm]
{\Huge \bfseries 정진학원 주간보고  \\[0.5cm] \Large 고등부 수학}\\[0.4cm] % Degree
\HRule \\[1.5cm]
\end{center}
}
\author{\LARGE 김태근 \\  \\[2cm]} % Your name and email address
\date{\today} % Beginning date
\maketitle

\tableofcontents

\mainmatter % Use Arabic numerals for page numbers

%----------------------------------------------------------------------------------------
%	LAB BOOK CONTENTS
%----------------------------------------------------------------------------------------

% Blank template to use for new days:

%\labday{Day, Date Month Year}

%\experiment{}

%Text

%-----------------------------------------

%\experiment{}

%Text

%----------------------------------------------------------------------------------------

\labday{2016.08.11 $-$ 2016.08.18}

\experiment{tb}

\begin{table}[h]
\centering
\begin{adjustbox}{width=.9\textwidth}
\begin{tabular}{c||c|c}
\toprule
\midrule
학년 & \multicolumn{2}{c}{교재 목록} \\
\hhline{=||==}
고1 & 선행 & 개념유형 수학2 \& 메시지 수학2 \\
\hline
고2 & 선행 & 개념유형 미적분1 \& 메시지 미적분1\\
\hline
\multirow{1}{*}{고3} & 공통 & 수능특강 수학2 \& 미적분1 (완료), 확률과통계 \\

\hline
\end{tabular}
\end{adjustbox}
\caption{\label{tab:i} 고등부 수학 교재목록 }
\end{table}

\experiment{cc}


\begin{table}[h]
\centering
\begin{adjustbox}{width=\textwidth}
\begin{tabular}{c||c|c}
\toprule
\midrule
학년 & \multicolumn{2}{c}{주간 진도 현황} \\
\hhline{=||==}
고1 & 선행 & 수학2 - 유리함수, 무리함수\\
\hline
고2 & 선행 & 미적분1 - 도함수의 활용 - 5. 그래프 $-$ 7. 속도와 가속도\\
\hline
\multirow{1}{*}{고3} & 수능 & 수능특강 수학2 - 적분, 적분의 활용, 확률과통계 - 순열, 조합, 확률 \\ 

\hline
\end{tabular}
\end{adjustbox}
\caption{\label{tab:ii} 고등부 수학 진도 현황 }
\end{table}

%-----------------------------------------

\experiment{at} % Multiple experiments can be included in a single day, this allows you to segment what was done each day into separate categories



\begin{table}[H]
\centering
\begin{adjustbox}{width=\textwidth}
\begin{tabular}{c|c||c|c|c|c|c}
\toprule
\midrule
학년 & 이름 & 결석 & 숙제불이행 & 교재미지참 & 수업태도 & 종합태도점수 \\
\hhline{=|=||=|=|=|=|=}
\multirow{4}{*}{고1}			& 김주희 & 0 & 2 & 0 & 5 & 90   \\ \hhline{~------}
					& 손예진 & 0 & 1 & 0 & 5 & 95  \\ \hhline{~------}
					& 이승현 & 0 & 0 & 0 & 5 & 100  \\ \hhline{~------}
					& 양진혁 & 0 & 2 & 0 & 5 & 90 \\
\hline
\multirow{1}{*}{고2}			& 안상호 & 0 & 1 & 0 & 5 & 95	\\ 
\hline
\multirow{2}{*}{고3}			& 김현우 & 0 & 0 & 0 & 5 & 100  \\ \hhline{~------}
					& 장한수 & 0 & 1 & 0 & 4 & 83	\\
\hline
\end{tabular}
\end{adjustbox}
\caption{\label{tab:iii} 고등부 태도 현황 }
\end{table}

%-----------------------------------------

\experiment{cm}
\begin{itemize}
	\item 고1은 승현이만이 숙제를 매우 성실히 잘해오고 주희와 진혁이는 미흡함. 이번 주말을 기점으로 다 해온다 했으니 지켜봐야 될듯.
	\item 이번 토요일엔 승현이가 한자시험관계로 결석하므로 주희와 진혁이를 데리고 시험을 볼 예정.
	\item 고2는 중간고사 범위까지 개념 공부는 끝났고 문제 푸는 중.
	\item 고3은 다음 주 내로 수능 특강을 마무리 짓고 모의고사를 한번 푼 후 9월 모의고사 예정.
\end{itemize}


%-----------------------------------------

\experiment{bo}
\begin{itemize}
 \item 고3: 
 \begin{itemize}
  \item 메가스터디 빅데이터 수학2 수능기출문제집 (13000) 2권 + 교사용 1권
  \item 메가스터디 빅데이터 미적분1 수능기출문제집 (13000) 2권 + 교사용 1권
  \item 메가스터디 빅데이터 확률과통계 수능기출문제집 (13000) 2권 + 교사용 1권
 \end{itemize}
 \item 고1:
 \begin{itemize}
  \item 마플 시너지 수학2 내신문제집 (20000원) 3권 + 교사용 1권
  \item 씨리얼 고등 수학영역 수학2 (451제) (12000원) 3권 + 교사용 1권
 \end{itemize}
\end{itemize}
(교사용이 없으면 학생용도 물론 괜찮습니다.) 

%----------------------------------------------------------------------------------------
%	FORMULAE AND MEDIA RECIPES
%----------------------------------------------------------------------------------------

------------------------------------------
%	MEDIA RECIPES
%----------------------------------------------------------------------------------------


%-----------------------------------------

%\textbf{Media 2}\\ \\

%Description

%----------------------------------------------------------------------------------------
%	FORMULAE
%----------------------------------------------------------------------------------------


%-----------------------------------------

%\textbf{Formula X - Description}\\ \\

%Formula

%----------------------------------------------------------------------------------------

\end{document}