%%%%%%%%%%%%%%%%%%%%%%%%%%%%%%%%%%%%%%%%%
% Short Sectioned Assignment
% LaTeX Template
% Version 1.0 (5/5/12)
%
% This template has been downloaded from:
% http://www.LaTeXTemplates.com
%
% Original author:
% Frits Wenneker (http://www.howtotex.com)
%
% License:
% CC BY-NC-SA 3.0 (http://creativecommons.org/licenses/by-nc-sa/3.0/)
%
%%%%%%%%%%%%%%%%%%%%%%%%%%%%%%%%%%%%%%%%%

%----------------------------------------------------------------------------------------
%	PACKAGES AND OTHER DOCUMENT CONFIGURATIONS
%----------------------------------------------------------------------------------------

\documentclass[paper=a4, fontsize=11pt]{scrartcl} % A4 paper and 11pt font size

\usepackage[T1]{fontenc} % Use 8-bit encoding that has 256 glyphs
\usepackage{fourier} % Use the Adobe Utopia font for the document - comment this line to return to the LaTeX default
\usepackage[english]{babel} % English language/hyphenation
\usepackage{amsmath,amsfonts,amsthm} % Math packages
\usepackage{amssymb}
\usepackage{kotex}
\usepackage{graphicx}
\usepackage{csquotes}
\usepackage[normalem]{ulem}
\usepackage{mdframed}
\usepackage{sectsty} % Allows customizing section commands
\allsectionsfont{\centering \normalfont\scshape} % Make all sections centered, the default font and small caps
\usepackage{tensor}
\usepackage{setspace}
\usepackage{twoopt}
\usepackage{fancyhdr} % Custom headers and footers
\pagestyle{fancyplain} % Makes all pages in the document conform to the custom headers and footers
\fancyhead{} % No page header - if you want one, create it in the same way as the footers below
\fancyfoot[L]{} % Empty left footer
\fancyfoot[C]{} % Empty center footer
\fancyfoot[R]{\thepage} % Page numbering for right footer
\renewcommand{\headrulewidth}{0pt} % Remove header underlines
\renewcommand{\footrulewidth}{0pt} % Remove footer underlines
\setlength{\headheight}{13.6pt} % Customize the height of the header

\numberwithin{equation}{section} % Number equations within sections (i.e. 1.1, 1.2, 2.1, 2.2 instead of 1, 2, 3, 4)
\numberwithin{figure}{section} % Number figures within sections (i.e. 1.1, 1.2, 2.1, 2.2 instead of 1, 2, 3, 4)
\numberwithin{table}{section} % Number tables within sections (i.e. 1.1, 1.2, 2.1, 2.2 instead of 1, 2, 3, 4)

\setlength\parindent{0pt} % Removes all indentation from paragraphs - comment this line for an assignment with lots of text

%----------------------------------------------------------------------------------------
%	DEFINITION & THEOREM
%----------------------------------------------------------------------------------------

\theoremstyle{plain}
\newmdtheoremenv{dfn}{Definition}[section]
\newcommand{\bigdef}[2]{\index{#1}\begin{dfn} {\rm #2} \end{dfn}}
\newcommand{\smalldef}[1]{\index{#1} {\em #1}}
\newmdtheoremenv{thm}[dfn]{Theorem}
\newmdtheoremenv{exmp}[dfn]{Example} % same for example numbers
\newmdtheoremenv{lem}[dfn]{Lemma}
\newmdtheoremenv{cor}[dfn]{Corollary}
\renewcommand{\qed}{\hfill $\Box$}
\renewenvironment{proof}{\par{\bf Proof:}}{\qed \par}

%----------------------------------------------------------------------------------------
%	BLANCK COMMAND
%----------------------------------------------------------------------------------------

\newcommand{\Com}{,\Hs}
\newcommand{\Hs}{\hspace{0.1cm}}
\newcommand{\HS}{\hspace{0.5cm}}
\newcommand{\Vs}{\vspace{0.1cm}}
\newcommand{\VS}{\vspace{0.3cm}}

%----------------------------------------------------------------------------------------
%	DIFFERENTIATE COMMAND
%----------------------------------------------------------------------------------------

\newcommand{\OD}[2]{\frac{d #1}{d #2}}
\newcommand{\PD}[2]{\frac{\partial #1}{\partial #2}}

%----------------------------------------------------------------------------------------
%	ETC COMMAND
%----------------------------------------------------------------------------------------

\newcommand{\BKS}[1]{\left( #1 \right)}

%----------------------------------------------------------------------------------------
%	MATH COMMAND
%----------------------------------------------------------------------------------------
\newcommand{\St}{\Hs such \Hs that \Hs}
\newcommand{\Nb}[1][A]{n(#1)}
\newcommand{\nin}{\notin}
\newcommand{\nsubset}{\not\subset}
\newcommand{\Set}[1]{\{ #1 \}}
\newcommand{\CSet}[2][x]{\{#1 | #2 \}}
\newcommand{\A}{$A$}
\newcommand{\B}{$B$}
\newcommand{\N}{\mathbb{N}}
\newcommand{\Z}{\mathbb{Z}}
\newcommand{\Q}{\mathbb{Q}}
\newcommand{\R}{\mathbb{R}}
\newcommand{\C}{\mathbb{C}}
\newcommandtwoopt{\Commutator}[2][X][Y]{\left[ #1 \Com #2 \right] }
\newcommandtwoopt{\RCommutator}[3][X][Y]{#1\BKS{#2(#3)} - #2\BKS{#1(#3)}}
\newcommandtwoopt{\ECommutator}[2][X][Y]{\BKS{#1^\nu \PD{#2^\mu}{x^\nu} - #2^\nu \PD{#1^\mu}{x^\nu}}}

%----------------------------------------------------------------------------------------
%	TITLE SECTION
%----------------------------------------------------------------------------------------

\newcommand{\horrule}[1]{\rule{\linewidth}{#1}} % Create horizontal rule command with 1 argument of height

\title{	
\normalfont \normalsize 
\textsc{Deparment of physics, Yonsei University} \\ [25pt] % Your university, school and/or department name(s)
\horrule{0.5pt} \\[0.4cm] % Thin top horizontal rule
\huge Advanced Mathematics 2\\ % The assignment title
\horrule{2pt} \\[0.5cm] % Thick bottom horizontal rule
}

\author{Axect} % Your name

\date{\normalsize\today} % Today's date or a custom date

\begin{document}

\maketitle % Print the title

%----------------------------------------------------------------------------------------
%	SECTION 0
%----------------------------------------------------------------------------------------

\section{Basic Notation}
\VS\VS
\begin{itemize}
 \item $\N$ : Natural number (자연수), $\Z$ : Integer (정수), $\Q$ : Rational number (유리수), \newline
 $\R$ : Real number (실수), $\C$ : Complex number (복소수)
 \item $\forall$ : For any (모든) \HS ex) $\forall x \in \N \Com x>0$
 \item $\exists$ : Exist (존재하다) \HS ex) $\exists x \in \N \Hs such \Hs that \Hs x>10$
 \item $\nexists$ : Not exist (존재하지 않는) \HS ex) $\nexists x \in \N \Hs such \Hs that\Hs x \leq 0$
 \item $\exists !$ : Uniquely exist (하나만 존재) \HS ex) $\exists ! x \in \N \Hs such \Hs that \Hs x \leq 1$
 \item $\alpha\Com\beta\Com\gamma$ : Alpha, Beta, Gamma (다 알겠죠?)
 \item $\phi$ : Phi (파이)$ -$ 집합에서는 공집합으로 쓰임. ($\pi$ (=Pi) 와는 다름)
 \item $\theta$ : Theta (세타) $-$ 보통 각도로 많이 쓰임.
 \item $\omega$ : Omega (오메가) $-$ \Hs $x^3 = \pm 1$ 의 허근이나 각도의 변화율을 나타낼 때 쓰임.
 \item $\Sigma$ : Sigma (시그마) $-$ 시그마의 대문자. 수학에서 수열의 합을 의미함.
 \item \textbf{Definition} : 정의 $-$ 받아들여야 하는 수학적 사실.
 \item \textbf{Theorem} : 정리 $-$ 정의로부터 도출가능한 명제.
 \item \textbf{Corollary} : 따름정리 $-$ 정리로부터 즉시 도출되는 명제. 
\end{itemize}

\pagebreak
\section{Vacation Test 1}
\VS

1. 두 집합 $A=\CSet{|x-1|>3}\Com B=\CSet{(x+3)(x-a)(x-a-1)(x-5) \leq 0}$ 에 대하여 $A\cap B$의 정수해가 4개일 때의
$a$값을 원소로 가지는 집합 $C$를 조건제시법으로 나타내시오.

\vspace{5cm}

2. 두 집합 $A=\CSet{|x-1| \leq p}\Com B=\CSet{x^2 + ax + 10 \geq 0 \Com a\in A}$ 에 대하여 $B = \R$ 일 때, 정수 $p$의 최댓값을 구하여라.

\vspace{3cm}

3. 두 집합 $A=\CSet[(x,y)]{(x-1)^2 + y^2 \leq 4}\Com B=\CSet[(x,y)]{|x-1|+|y| \leq a}$ 에 대하여 $A\subset B$이기 위한 양수 $a$의 최솟값은?

\vspace{3cm}

4. 기본적으로 집합 \A 에 대하여 $A \nsubset A^c$ 이다. 그러나 $A \nsubset A^c$ 를 만족하는 집합 \A가 존재하는데 이때, 그 예를 적고 이유를 설명하시오.

\vspace{3cm}

5.전체집합 $U$의 두 부분집합 \A, \B에 대하여 $A\cup B^c = B^c$ 일 때, $(A-B)\cup (B-A)$ 를 최대한 간단히 나타내시오.

\pagebreak
%----------------------------------------------------------------------------------------
%	PROBLEM 2
%----------------------------------------------------------------------------------------
6. 전체집합 $U=\Set{1,2,\cdots ,10}$의 부분집합 $A={1,2,3,4,5,6,7}$에 대하여 $(A-X)\cup X = A$ 일 때, 집합 $X$의 개수는? (단, $X\neq \phi$)

\vspace{3cm}

7. 전체집합 $U=\CSet{x \leq 10 \Com x\in \N}$ 의 두 부분집합 $A=\Set{1,2,\cdots ,7}\Com B=\Set{2,4,6}$에 대하여 \linebreak $A\cap X = B \cup X$ 를 만족시키는 집합 $X$의 개수는?

\vspace{3cm}

8. 어느 고등학교의 학생 300명을 대상으로 뮌헨, 프라하, 이스탄불 중 가고 싶은 곳을 선택하는 설문조사를 하였더니 다음과 같은 사실을 알게 되었다.

\begin{enumerate}
 \item 이스탄불을 선택한 학생들은 프라하도 선택하였다.
 \item 이스탄불을 선택하지 않은 학생 중에 뮌헨과 프라하를 모두 선택한 학생들은 30명이다.
 \item 뮌헨과 이스탄불을 동시에 선택하지 않은 학생은 260명이다.
 \item 아무 곳도 선택하지 않은 학생은 20명이다.
\end{enumerate}

이때, 뮌헨을 선택한 학생수를 $a$, 프라하를 선택한 학생 수를 $b$라 하면 $a+b$는 얼마인가?

\vspace{5cm}

9. 전체집합 $U=\CSet[2k-1]{k \leq 6 \Com k\in \N}$ 의 두 부분집합 $A=\Set{1,3,5,7}\Com B=\Set{5,7,9,11}$에 대하여 집합 $(B-A)^c -B^c$의 모든 원소의 합을
구하시오.

\pagebreak


%------------------------------------------------
10. 실수 $x$에 대하여 두 조건 $p,q$가 다음과 같다. 

\begin{equation*}
 p\Hs :\Hs |x|\leq 4\Com q\Hs :\Hs x^2 -4x-12\leq 0
\end{equation*}

조건 '$p$이고 $\sim q$' 의 진리집합에 속하는 정수의 개수는?

\vspace{3cm}

11. 명제 '모든 실수 $x$에 대하여 $2x^2 -2kx+3k>0$이다' 가 거짓이 되도록 하는 자연수 $k$의 최솟값은?

\vspace{2.5cm}

12. $x,y \in \R$일 떄, 다음 명제 중 역은 참이고 대우는 거짓인 명제인 것만을 있는 대로 고르면?

\begin{enumerate}
 \item $x^2 =1$ 이면 $x=1$이다.
 \item $x>y$이면 $x^2 > y^2$이다.
 \item $x$가 무리수이면 $x^2$도 무리수이다.
\end{enumerate}

\vspace{2.5cm}

13. $\sqrt{3}$이 무리수임을 증명하여라.

\pagebreak
%------------------------------------------------
14. 모든 실수 $x,y$에 대하여 부등식 $x^2 + |y-5| +10x+k>0$ 이 성립하도록 하는 자연수 $k$의 최솟값을 구하시오.

\vspace{3cm}

15. 양수 $a$에 대하여 $a + \frac{9}{a+1}$는 $a=p$일 때, 최솟값 $q$를 갖는다. $p+q$의 값을 구하시오.

\vspace{3cm}

16. 두 실수 $a,b$에 대하여 조건 $p\Hs:\Hs|a|+|b|>0$이 있다. 다음 중 $\sim p$이기 위한 필요충분조건인 것은?

\begin{enumerate}
 \item $ab=0$
 \item $a+b=0$
 \item $a^3 +b^3 =0 $
 \item $a + b \sqrt{2} = 0$
 \item $a+b\sqrt{-1} = 0 $
\end{enumerate}

\VS

17. $x$가 양수일 때, 세 점 $P\BKS{x, \frac{1}{x}}\Com Q\BKS{-1,0}\Com R\BKS{0,-9}$를 꼭짓점으로 하는 삼각형 $PQR$의 넓이는 $x=p$일 때, 최솟값
$q$를 갖는다. $p\times q$의 값은?

\pagebreak
%----------------------------------------------------------------------------------------
\paragraph{$-$ Essay}

다음 제시문을 읽고 물음에 답하시오.

\begin{enumerate}
 \item 원의 방정식 $x^2 +y^2 =r^2$은 매개변수를 이용하여 나타낼 수 있다. 간단하게 식만 보면 어떤 두 수를 제곱하여 더하면 일정한 값이 나온다는 것인데,
 우리는 이미 이런 값들을 알고 있다. 바로 삼각비인 $\sin$과 $\cos$이다. $\sin^2{\theta} + \cos^2{\theta} = 1$ 이므로 $x = r\cos{\theta}\Com y=r\sin{\theta}$
 로 치환하면 원의 방정식을 만족한다. 즉, 기존의 직교좌표계가 $(r,\theta)$ 좌표계로 바뀌는데 이를 극좌표계라고 부른다.
 \item 평균변화율의 정의는 $\frac{\Delta y}{\Delta x}$인데, 이는 시작점을 $(x_0 , y_0)$라 두고 끝 점을 $(x,y)$라 두면 $\frac{y-y_0}{x-x_0}$가 된다.
 순간변화율의 정의는 미분이라는 연산으로 표현되는데 이는 접선의 기울기와 같다. 평균변화율과 순간변화율이 항상 일정한 도형을 직선이라고 부르고 이때의 
 평균변화율 혹은 순간변화율을 직선의 기울기라고 부른다.
 \item 원과 직선의 위치관계를 판별하는 방법은 크게 두 가지가 있다. 하나는 원의 방정식과 직선의 방정식을 연립하여 한 문자를 소거한 뒤 판별식을 사용하는 방법이 있고,
 다른 하나는 원의 중심과 직선까지의 거리 공식을 사용하는 방법이 있다. 이때, 점과 직선사이의 거리 공식은 다음과 같다. (점 $(x_1,y_1)$과 $ax+by+c=0$의 거리)
 \begin{equation*}
  d = \frac{|ax_1+by_1+c|}{\sqrt{x^2 + y^2}}
 \end{equation*}
\end{enumerate}
\VS
18. \Hs 제시문 1.에서 사용했던 방법을 거꾸로 이용하여 다음 식을 직교좌표계로 변형시키시오.
\begin{equation*}
 1+\tan^2{\theta} = \frac{p}{\cos{\theta}} + \frac{q}{\cos^2{\theta}}
\end{equation*}

\vspace{2cm}

19. 18번에서 구한 도형에 $p=2\Com q=15$을 대입했을 때, 도형이 어떤 도형인지 설명하시오.

\vspace{3cm}

20. 모든 제시문을 사용하여 다음 식의 최솟값과 최댓값을 구하시오.
\begin{equation*}
 \frac{\sin x + 2}{\cos x + 3}
\end{equation*}

%----------------------------------------------------------------------------------------

\end{document}