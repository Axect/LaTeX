%%%%%%%%%%%%%%% SOCIALST %%%%%%%%%%%%%%%%%%
%%% COMMAND by Socialst.sty             %%%
%%% style by Tae Geun Kim               %%%
%%%%%%%%%%%%%%%%%%%%%%%%%%%%%%%%%%%%%%%%%%%

%%%%%%%%%%%%%%% CLASS %%%%%%%%%%%%%%%%%%%%%
\documentclass[%
 reprint,
 amsmath,amssymb,
 aps,
]{revtex4-1}
%%%%%%%%%%%%%%% PACKAGE %%%%%%%%%%%%%%%%%%%
\usepackage{Socialst}
%%%%%%%%%%%%%%% DOCUMENTS %%%%%%%%%%%%%%%%%
\begin{document}

% ------------- TITLE ---------------------
\title{Stepwise General Relativity Calculations with CAS \\
  {\NM Using SageManifolds}
}

% ------------- AUTHOR --------------------
\author{Tae Geun Kim}
\affiliation{Department of Physics, Yonsei University}
\email{edeftg@gmail.com}

% ------------- DATE ----------------------
\date{\today}

% ------------- ABSTRACT ------------------
\begin{abstract}
  General Relativity (GR) is one of main columns in modern physics.
  Its great geometrical structure not only provides beauty but also
  offers extremely accurate calculation results.
  But since its nonlinearity, it's too painful to calculate with hands. 
  However, there is a great solution to overcome it - Computer Algebra System (CAS).
  CAS automatically calculates all the variables of the GR when you insert only the initial
  conditions. Although there are some good CAS programs now such as Mathematica, Maple, Maxima,
  Sympy, SageManifolds and etc, some of them are too expensive to students
  or not too rigorous to do something with Geometry. SageManifolds can be good choice
  to solve these problems. In this paper, I'll use SageManifolds to calculate
  geometric objects in General Relativity for specific case -
  Friedmann-Robertson-Walker Metric. 
\end{abstract}

% ------------- MAKE TITLE ----------------
\maketitle
%\tableofcontents

% ------------- INTRO ---------------------
\section{\label{sec:intro}Introduction}

\subsection{Necessity of CAS}
There are two greatest accurate calculations by human being - Quantum
Electrodynamics \& General Relativity (GR). In particular, GR had become a
representative of elegant physics due to its structural beauty.
We can represent GR with just one equation:
\begin{equation}
  \tensor{G}{_\mu_\nu} + \Lambda g_{\mu\nu}= 8\pi G_N \tensor{T}{_\mu_\nu}
\end{equation}
This short equation can explain the origin of gravity, blackhole, gravitational
wave and even cosmic expansion.
Like many other fields of physics, its computation procedure is mechanistic.
Thus, almost all of GR computing procedures are same except input variables, for
example, metric tensor. However the process from metric to Einstein equation is
continuation of hardship and adversity. To get our final goal, Einstein tensor,
we must go through a number of intermediate steps, including connection
coefficients and Riemann tensor. In addition, each process must undergo very
complex calculations. Because of this complexity, we must be helped by someone
who can easily do these calculations. In the past, if such a person was a
mathematician or graduate student capable of calculating, it is now a computer
program. And especially in this case, Computer Algebra System (CAS) is the best
choice. Now, there are a lot of great CAS such as Mathematica, Maple, Maxima,
Sympy and Sagemath. There is a good reference comparing these systems
\cite{symbol}.

\subsection{SageManifolds}

As mentioned above, there are many CASs now. Typical CASs have internal packages
for general relativity calculations. For example, Mathematica has GR, Ricci
packages, Maxima has ctensor package and Sagemath has SageManifolds. You can
choose one of them to compute GR objects attending to your preference.  I choose
the SageManifolds because it allows for
rigorous differential geometry calculations. Unusually, SageManifolds deals with
differentiable manifolds of arbitrary dimension not just a tensor calculation.
Thus in this system, a tensor field is then discribed by its representations in
each vector frame. It contains lots of functionalities such as topological
manifolds, exterior calculus, Lie derivatives of tensor fields, general affine
connections and some plotting capabilities \cite{sage}. 

Due to its unique structure, SageManifolds has clear advantages and
disadvantages.
Although Sagemath is user friendly and convenient to use, it's not. You can't
obtain any tensor calculations unless you are good at differential geometrical
structure. So if this is your first time to use SageManifolds, then you will be
confused by so many errors. But if you are good at differential geometry and
adapt to using SageManifolds, you will be able to do graceful and effective
computation.

\section{Progress}

To obtain physical equation from General Relativity, we should need Einstein
equation. In order to get the Einstein equation, it's necessary to determine
geometrical structure and obtain Energy-Momentum tensor. So, let's start to
determine geometrical structures with SageManifolds first.

\subsection{Geometrical Structure}

The differential geometry is based on differentiable manifolds. So, we should
start with defining manifolds.
\begin{lstlisting}
  M = Manifold(4, 'M', '\mathcal{M}')
\end{lstlisting}
The '{\ttfamily Manifold}' function receive dimension, name and latex name. We
want to deal with FRW metric in the Minkowski spacetime, so we should input
dimension 4. Then now, we should define chart (coordinate).
\begin{lstlisting}
  fr.<t,r,th,ph> = M.chart(r't r:[0,+oo) th:[0,pi]:\theta ph:[0,2*pi):\phi')
\end{lstlisting}
The `\code{fr.<t,r,th,ph>}' defines `\code{fr}'as chart which has coordinate
$(t,r,\theta,\phi)$. RHS is a little bit complex. I gave range of each
coordinates through that code. Don't insert blank in range. If you do that, you
can see error.

We defined manifold and chart. To play with GR, we should insert metric tensor.
We use Friedmann-Robertson-Walker metric as follows.
\begin{equation}
  g_{ab} = -dt^2 + a(t)^2 \left[ \frac{dr^2}{1-Kr^2} + r^2(d\theta^2  + \sin^2(\theta)d\phi^2)\right]
\end{equation}
Ofcourse, We can use this metric tensor in SageManifolds.
\begin{lstlisting}
  # Declare Variables
  var('Lambda K G', domain='real')
  a = M.scalar_field(function('a')(t), name='a')

  # Define Metric
  g = M.lorentzian_metric('g')
  g[0,0] = -1
  g[1,1] = a^2/(1-K*r^2)
  g[2,2] = a^2*r^2
  g[3,3] = a^2*r^2*sin(th)^2
\end{lstlisting}  
Code is simple. There were some variables except coordinates such as
$K,\,G,\,\Lambda$ in FRW cosmology, thus we should declare those variables
first. And trivially FRW theory allows only real variables, so should check it.
For metric, we usually use Lorentzian metric not Riemannian. SageManifolds has
both as sub-objects in \code{Manifold}, so you can choose it for each theory.
Since we defined 4-dimensional \code{Manifold}, so automatically our metric
tensor has 16 components. In SageManifolds, default component value is 0, so we
just input diagonal componets in this case. Putting componets is very simple.
Just type.

Now, we have metric tensor. It means we can do anything in General Relativity in
principle. But only with hands, just taking all connection components is too
hard to do. However, it's too easy in SageManifolds.

\begin{lstlisting}
  nab = g.connection()
  show(nab.display(only_nonredundant=true))
\end{lstlisting}
Because in SageManifolds, metric already has connection method. Just type
\code{g.connection()}, that's it. You can see the result by `display' method.
`\code{only\_nonredundant=true}' means just display non vanishing components by
symmetry. 
\begin{eqnarray*}
  \Gamma_{ \phantom{\, t} \, r \, r }^{ \, t \phantom{\, r}
  \phantom{\, r} } & = & -\frac{a\left(t\right) \frac{\partial\,a}{\partial t}}{K r^{2} - 1} \\
  \Gamma_{ \phantom{\, t} \, {\theta} \, {\theta} }^{ \, t \phantom{\, {\theta}}
  \phantom{\, {\theta}} } & = & r^{2} a\left(t\right)
                                \frac{\partial\,a}{\partial t} \\
  \Gamma_{ \phantom{\, t} \, {\phi} \, {\phi} }^{ \, t \phantom{\, {\phi}}
  \phantom{\, {\phi}} } & = & r^{2} a\left(t\right) \sin\left({\theta}\right)^{2}
                              \frac{\partial\,a}{\partial t} \\
  \Gamma_{\phantom{\, r} \, t \, r }^{ \, r \phantom{\, t} \phantom{\, r} } & =
                       & \frac{\frac{\partial\,a}{\partial t}}{a\left(t\right)}
  \\
  \Gamma_{ \phantom{\, r} \, r \, r }^{ \, r \phantom{\, r} \phantom{\, r} } & =
                       & -\frac{K r}{K r^{2} - 1} \\
  \Gamma_{ \phantom{\, r} \, {\theta} \, {\theta} }^{ \, r \phantom{\, {\theta}}
  \phantom{\, {\theta}} } & = & K r^{3} - r \\
  \Gamma_{ \phantom{\, r} \, {\phi} \, {\phi} }^{ \, r \phantom{\, {\phi}}
  \phantom{\, {\phi}} } & = & {\left(K r^{3} - r\right)}
                              \sin\left({\theta}\right)^{2} \\
  \Gamma_{ \phantom{\, {\theta}} \, t \, {\theta} }^{ \, {\theta} \phantom{\, t}
  \phantom{\, {\theta}} } & = & \frac{\frac{\partial\,a}{\partial
                                t}}{a\left(t\right)} \\
  \Gamma_{ \phantom{\, {\theta}} \, r \, {\theta} }^{ \, {\theta} \phantom{\, r}
  \phantom{\, {\theta}} } & = & \frac{1}{r} \\
  \Gamma_{ \phantom{\, {\theta}} \, {\phi} \, {\phi} }^{ \, {\theta} \phantom{\,
  {\phi}} \phantom{\, {\phi}} } & = & -\cos\left({\theta}\right)
                                      \sin\left({\theta}\right) \\
  \Gamma_{ \phantom{\, {\phi}} \, t \, {\phi} }^{ \, {\phi} \phantom{\, t}
  \phantom{\, {\phi}} } & = & \frac{\frac{\partial\,a}{\partial
                              t}}{a\left(t\right)} \\
  \Gamma_{ \phantom{\, {\phi}} \, r \, {\phi} }^{ \, {\phi} \phantom{\, r} \phantom{\, {\phi}} } & = & \frac{1}{r} \\
  \Gamma_{ \phantom{\, {\phi}} \, {\theta} \, {\phi} }^{ \, {\phi} \phantom{\,
  {\theta}} \phantom{\, {\phi}} } & = &
                                        \frac{\cos\left({\theta}\right)}{\sin\left({\theta}\right)}
\end{eqnarray*}
This is the result of above code. SageManifolds has latex auto-lendering
feature, so you can get beautiful results or latex code directly. This feature
is so helpful to write articles.

You can also check that this is torsion free and see really the metric is
parallel transported with the following code:
\begin{lstlisting}
  show(nab.torsion().display())
  show(nab(g).display())
\end{lstlisting}

The main object of geometric part of GR is curvature. Now, let's see how to deal
with it in SageManifolds.

\begin{lstlisting}
  Ric = nab.ricci()
  show(Ric.display())
\end{lstlisting}
In SageManifolds, connection \& metric have `\code{ricci()}' method both. You
can get Ricci tensor from just a line of code.

\newpage

\begin{equation}
  \begin{aligned}
    \mathrm{Ric} &\left(g\right) = -\frac{3 \, \frac{\partial^2\,a}{\partial t ^ 2}}{a\left(t\right)} \mathrm{d} t\otimes \mathrm{d} t\\
    &+ \left( -\frac{2 \, \left(\frac{\partial\,a}{\partial t}\right)^{2} + a\left(t\right) \frac{\partial^2\,a}{\partial t ^ 2} + 2 \, K}{K r^{2} - 1} \right) \mathrm{d} r\otimes \mathrm{d} r \\
    &+ \left( 2 \, r^{2} \left(\frac{\partial\,a}{\partial t}\right)^{2} + r^{2} a\left(t\right) \frac{\partial^2\,a}{\partial t ^ 2} + 2 \, K r^{2} \right) \mathrm{d} {\theta}\otimes \mathrm{d} {\theta}\\
    &+ {\left(2 \, r^{2} \left(\frac{\partial\,a}{\partial t}\right)^{2} + r^{2} a\left(t\right) \frac{\partial^2\,a}{\partial t ^ 2} + 2 \, K r^{2}\right)} \sin\left({\theta}\right)^{2} \\
    &\HS\HS\mathrm{d} {\phi}\otimes \mathrm{d} {\phi}
  \end{aligned}
\end{equation}

\vs

Except internal method, SageManifolds provides index product (contraction or
tensor product). So, we can use it to obtain Ricci scalar.

\begin{lstlisting}
  R = g.inverse()['^ab']*Ric['_ab']
\end{lstlisting}
`\code{<Tensor>['\_ab']}' or `\code{<Tensor>['\textasciicircum ab']}' means
$T_{ab}$ or $T^{ab}$. And we can implement either contraction or tensor products
simply by multiplying them.

Finally, we can get Einstein tensor.

\begin{lstlisting}
  Gab = Ric - 1/2*R*g
  Gab.set_name(r'G_{ab}')
  show(Gab.display())
\end{lstlisting}

\begin{equation}
  \begin{aligned}
  G_{ab} &= \frac{3 \, {\left(\left(\frac{\partial\,a}{\partial t}\right)^{2} + K\right)}}{a\left(t\right)^{2}} \mathrm{d} t\otimes \mathrm{d} t\\
  &+ \left( \frac{\left(\frac{\partial\,a}{\partial t}\right)^{2} + 2 \, a\left(t\right) \frac{\partial^2\,a}{\partial t ^ 2} + K}{K r^{2} - 1} \right) \mathrm{d} r\otimes \mathrm{d} r\\
  &+ \left( -r^{2} \left(\frac{\partial\,a}{\partial t}\right)^{2} - 2 \, r^{2} a\left(t\right) \frac{\partial^2\,a}{\partial t ^ 2} - K r^{2} \right) \mathrm{d} {\theta}\otimes \mathrm{d} {\theta}\\
  &-{\left(r^{2} \left(\frac{\partial\,a}{\partial t}\right)^{2} + 2 \, r^{2} a\left(t\right) \frac{\partial^2\,a}{\partial t ^ 2} + K r^{2}\right)} \sin\left({\theta}\right)^{2} \\
  &\HS\HS\mathrm{d} {\phi}\otimes \mathrm{d} {\phi}
  \end{aligned}
\end{equation}

\subsection{Matter Information}

According to GR, to generate gravity, we should need matter which is source of
gravity. Inserting matter information consists of the following steps.

\begin{enumerate}[i)]
\item Write matter Lagrangian. (by hand)
\item Obtain Energy-Momentum tensor with Noether's thm.
\item Set variables except coordinate variables.
\item Insert Energy-Momentum tensor with variables.
\end{enumerate}

In case of FRW cosmology, there are well defined Energy-Momentum tensor as known
as perfect fluid.
\begin{equation}
  T_{ab} = (\rho + P)u_a u_b + Pg_{ab}
\end{equation}
So, let's start from third step.

\begin{lstlisting}
  # Declare scala fields
  rho = M.scalar_field(function('rho')(t), name='rho', latex_name=r'\rho')
  P = M.scalar_field(function('P')(t), name='P')

  # Declare Vector Fields
  u = M.vector_field('u', latex_name=r'u^a')
  u[0] = 1

  # Vector to Covector (1-form)
  u_form = u.down(g)
  u_form.set_name(r'u_a')

  # Define energy momentum tensor
  T = (rho + P)*u_form*u_form + P*g
  T.set_name(r'T_{ab}')
  show(T[:])
\end{lstlisting}
SageManifolds also allows matrix representation of tensor. `\code{show(T[:])}'
shows all components of tensor $T$ as matrix. 

\begin{equation}
  \left(\begin{array}{rrrr}
\rho\left(t\right) & 0 & 0 & 0 \\
0 & -\frac{P\left(t\right) a\left(t\right)^{2}}{K r^{2} - 1} & 0 & 0 \\
0 & 0 & r^{2} P\left(t\right) a\left(t\right)^{2} & 0 \\
0 & 0 & 0 & r^{2} P\left(t\right) a\left(t\right)^{2} \sin\left({\theta}\right)^{2}
\end{array}\right)
\end{equation}
And we can raise or lower indices with metric or inverse metric tensor.
\begin{lstlisting}
  Tmix = g.inverse()['^ab']*T['_bc']
  Tmix.set_name(r'{T^a}_b')
  show(Tmix.display())
  show(Tmix[:])
\end{lstlisting}

\begin{equation}
  \tensor{T}{^a_b} = \left(\begin{array}{rrrr}
-\rho\left(t\right) & 0 & 0 & 0 \\
0 & P\left(t\right) & 0 & 0 \\
0 & 0 & P\left(t\right) & 0 \\
0 & 0 & 0 & P\left(t\right)
\end{array}\right)
\end{equation}
We know Energy momentum tensor is conserved.

\begin{equation}
  \nabla_a\tensor{T}{^a_b} = 0
\end{equation}
Let's obtain this in SageManifolds.

\newpage

\begin{lstlisting}
  # Covariant Derivative of T
  DTmix = nab(Tmix)
  DTmix.set_name(r'\nabla T')

  # Contract = Energy-Momentum Conserved
  CT = DTmix['^a_ba']
  print("Energy-Momentum Conserved:\n")
  show(-CT[0].expr().expand() == 0)
\end{lstlisting}
\begin{equation}
  \frac{3 \, P\left(t\right) \frac{\partial}{\partial t}a\left(t\right)}{a\left(t\right)} + \frac{3 \, \rho\left(t\right) \frac{\partial}{\partial t}a\left(t\right)}{a\left(t\right)} + \frac{\partial}{\partial t}\rho\left(t\right) = 0
\end{equation}

\subsection{Construct Einstein Equation}

Now, we have Einstein tensor \& Energy Momentum tensor. So, it's easy to
construct Einstein equation. Typical Einstein Equation is given as:

\begin{equation}
  G_{ab} + \Lambda g_{ab} = 8\pi GT_{ab}
\end{equation}

In SageManifolds, just type next code, we can get complete form of Einstein
equation. 
\begin{lstlisting}
  EE = Gab + Lambda*g - 8*pi*G*T
  show(EE[0,0].expr().expand() == 0)
\end{lstlisting}

\begin{equation}
  -8 \, \pi G \rho\left(t\right) - \Lambda + \frac{3 \, \frac{\partial}{\partial t}a\left(t\right)^{2}}{a\left(t\right)^{2}} + \frac{3 \, K}{a\left(t\right)^{2}} = 0
\end{equation}
This equation is called Friedmann equation which is main equation in FRW cosmology.

\section{Conclusion}

So far, we have looked at the calculation process of general relativity theory
with SageManifolds. Maybe it's too basic example for someone, but it's enough to
show effectiveness of SageManifolds. Sagemath which includes SageManifolds is
the largest open source project for Computer Algebra System in the world. It
contains sympy and uses python syntax. Unlike some programs like Mathematica,
Sage project not only enables symbolic programming but also implements
mathematical structures such as geometry, algebra, and topology on a computer.
If you are engaged in academics related to mathematics, Sagemath will surely be
an excellent choice.

Especially in physics, Sagemath has a tremendous advantage,
the ability to get analytic expression for most well-know physics without
additional cost. For general relativity, we can obtain the differential equation
by simply substituting the metric tensor. Then you can solve that equation
numerically by your prefer programs. This program, called Sagemath, not only
gives us another tool to validate calculations, but also naturally leads to
rigidity and helps to reduce research time effectively.

\VS


\section{TODO}

My original purpose was to implement a mathematical structure of general
relativity using a functional programming language like Haskell or Scala, but I
could not implement it because of the short time. Therefore, after the end of the
semester, I would like to create the framework that automatically computes the
connection coefficient and Einstein equation by simply inputting the metric
tensor and then uses it to perform numerical computations. This work is expected
to be of great significance since it will surely be useful for future research. 

\appendix

\bibliography{../../BibTeX/ref}

\end{document}
