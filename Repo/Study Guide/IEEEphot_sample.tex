\documentclass[final]{IEEEphot}

\usepackage{kotex}
\usepackage{setspace}
\renewcommand{\baselinestretch}{1.2} 
\jvol{xx}
\jnum{xx}
\jmonth{August}
\pubyear{2016}



\newtheorem{theorem}{Theorem}
\newtheorem{lemma}{Lemma}

\begin{document}

\title{Study Guide for Highschool Students}

\author{Tae-Geun, Kim}

\affil{\textbf{B.S.}: Dept of Astronomy, Yonsei University \\
\textbf{M.S.}: Dept of Physics, Yonsei University}

\maketitle


\begin{abstract}
 공부, 특히 대입을 위한 고등학교 입시 공부에는 철저한 계획과 전략이 있어야 합니다. 그러나 대개의 경우, 시험기간에는 계획을 좀 세워서 공부를 하지만,
 그 외에는 막상 학원에서 낸 숙제, 학교 수행평가 등을 '처리'할 뿐, 별다른 공부계획을 세우지 않습니다. 따라서 이 문서에서는 앞으로의 공부계획에 대하여
 필자의 경험을 토대로 대략적인 Guide line을 소개할 것입니다.
\end{abstract}


\section{Introduction}

\hspace{0.3cm} 입시는 크게 수시, 정시로 나누고 그 안에서 또 세부분류를 하게 됩니다. 수시의 경우, 내신으로 대표되는 학생부 일반 혹은 우수자 전형이 있고, 내신과 생활기록부로 평가하는
학생부 종합전형, 상위권 대학에 주로 존재하는 논술우수자 전형, 중상위권 대학에 존재하는 적성우수자 전형이 있습니다. 정시의 경우는 예체능의 경우를 제외하면
대부분 수능점수로 선발하는 일반전형입니다. 즉, 특수한 경우가 아닌 이상은 각 대학마다 입학할 수 있는 방법이 4가지가 있습니다. 이해하기 쉽게 연세대학교로 예를 들어 봅시다.

\begin{figure}[h]
\centering
 \includegraphics[height = 6cm, width = 14cm]{Yonsei.png}
\end{figure}

특기자 전형 978명을 제외하면, 정시 일반 전형이 1003명으로 38.2\%, 논술위주인 수시일반전형이 683명으로 26\%이며 학생부종합전형이 487명으로 18.5\%, 학생부교과전형이
257명으로 9.8\%이라는 것을 알 수 있습니다. 즉, 만약 연세대학교를 지원하고 싶다면 가장 넓은 길은 수능 위주인 정시일반전형으로 지원하는 것이고 가장 좁은 길은 학생부 위주인 교과전형이니
내신 공부보다는 수능공부를 열심히 하는 것이 좋은 방법일 것입니다. (놀랍게도 거의 대부분의 학교가 이렇습니다.)


\pagebreak


따라서 학생들은 목표대학을 설정하고, 그 학교에 적절한 전형을 탐색하고 그 중 제일 자신에게 적합한 전형을 골라 미리 대비해야 합니다. 그러나 많은 경우, 고1,2 때는 객관적으로 이를 파악하지 못하고
고3이 돼서야 대비를 시작합니다. 그리고 늦은 대비는 대개 좋은 결과를 내지 못합니다. 여기서는 자신에게 맞는 전형을 고르는 법과 각 전형별 대비하는 법, 그리고 공부계획을 설정하는 법 등에 대하여 일종의
Reference(참고문서)를 제시하고자 합니다. 그럼 이제부터 하나씩 찬찬히 살펴봅시다.

\section{전형 결정하기}

\hspace{0.3cm} 입시에서 가장 중요한 것은 물론 공부량이겠지만 어느 전형을 위주로 공부해야할지 결정하는 것도 큰 부분을 차지합니다. 내신이 좋지 못한데도 불구하고 학교에서 내신공부만 시킨다고 무작정
따라가거나, 모의고사 문제를 잘 못푸는데도 불구하고 수능대박을 노리면서 내신을 등한시하다가는 결국 자신의 목표에서 한참 뒤쳐진 대학을 갈 수 밖에 없습니다. 
그럼 전형을 어떻게 결정해야 할까요? 자신의 목표와 현재 내신등급을 보면 대략적으로 결정할 수 있습니다. 이를 위해서 내신등급을 다음과 같이 나누겠습니다.

\begin{itemize}
 \item \textbf{최상위권}: 내신 1등급 초반 (1 $-$ 1.5)
 \item \textbf{상위권}: 내신 1등급 후반 $-$ 2등급 초반
 \item \textbf{중상위권}: 내신 2등급 중반 $-$ 3등급 초반
 \item \textbf{중위권}: 내신 3점 중반 $-$ 내신 4점 초반 대
 \item \textbf{하위권}: 그 외 나머지.
\end{itemize}

목표대학 역시 나눠야합니다.

\begin{itemize}
 \item \textbf{서울 최상위}: 각종 의대, 서울대, 연세대, 고려대, 사관학교, 경찰대학 등
 \item \textbf{서울 상위}: 서강대, 성균관대, 한양대, 중앙대, 경희대, GIST, UNIST 등
 \item \textbf{서울 중상위}: 건국대, 홍익대, 동국대, 인하대, 부산대, 경북대 등
 \item \textbf{서울 하위}: 광운대, 명지대, 상명대, 숭실대, 국민대, 세종대, 인천대 등
 \item \textbf{경기권 상위}: 가천대, 단국대, 강남대, 전북대, 전남대, 충북대, 충남대 등
\end{itemize}

이제 분류를 대충 다 했으니 내신등급과 목표에 따라 대략적으로 전형을 정해봅시다.

\subsection{고등학교 1학년}

\hspace{0.3cm} 고등학교 1학년의 경우 선택의 폭이 굉장히 많습니다. 1학년 1학기 부터 3학년 1학기 까지 총 5개의 학기 중에서 아직 4개나 남았기 때문이죠. 그러나 특별한 경우가 아니라면, 자신의 등급이
확 뒤바뀌지는 않으므로 조금은 신중히 결정해야 합니다. 일단 대략적인 분류 기준은 다음과 같습니다.

\begin{itemize}
 \item \textbf{최상위권}: 모의고사 등급에 따라 다르겠지만 내신이 최상위이면 목표 대학에 따라 수능 최저등급을 알아보고 거기에 맞춰서 학생부교과전형으로 가는 것이 쉽습니다. \vspace{0.1cm}
 \item \textbf{상위권}: 상위권의 경우 내신등급이 서울 중상위권 수준입니다. 따라서 목표가 중상위면 계속 내신을 유지하며 최저를 맞추면 되지만, 목표가 상위 혹은 최상위라면 내신등급을 최상위권으로 끌어올려야 합니다. 
 혹은 모의고사 등급이 모든 과목 3등급 이내라면 (1등급 1개 정도 섞여 있는) 내신 등급을 유지하며 정시를 대비하는 것이 좀 더 쉬울 수 있습니다. \vspace{0.1cm}
 \item \textbf{중상위권}: 중상위권의 경우는 목표에 따라 심각하게 갈립니다. 목표가 서울 중상위라면 내신을 2학기부터 단계적으로 끌어올려 상위권에 진입하면 됩니다. 2학년 때 2점대를 달성하고, 3학년 1학기에 1점대가 나오면
 충분히 가능합니다. 그러나 목표가 상위 혹은 최상위라면 내신으로는 가망이 없습니다.\footnotemark 따라서 이 경우엔 내신대비는 줄이고 지금부터 정시 대비 체제로 가야합니다.
 (필자도 내신등급이 여기에 위치하였는데 1학년 때부터 꾸준히 정시준비를 했더니 3학년 끝난 후 내신은 상위권이 돼서 서울 중상위 수준이 되었으나 정시로는 최상위권을 진학할 수 있었습니다.) \vspace{0.1cm}
 \footnotetext{혹시 나는 되지 않을까? 라는 생각은 버리는 것이 좋습니다. 자신이 특별했다면 내신이 중상위일리도 없습니다.}
 \item \textbf{중위권}: 이 경우는 학교의 수준별로 차이가 납니다. 자신의 학교가 서울대를 5명 이상 보내는 꽤 좋은 학교라면 정시를 대비하는 것이 좋습니다. 그러나 그렇지 않다면, 내신 등급을 무작정 올려서 적어도
 내신 중상위권이 나오게 해야 하며 동시에 적성고사를 준비하여 경기권 상위를 노려보는 것이 좋습니다. \vspace{0.1cm}
 \item \textbf{하위권}: 이 경우는 철저하게 내신으로는 대학을 갈 수 없습니다. 전문대 역시 4등급이 넘어가면 합격가능성이 없습니다.\footnotemark 자사고의 경우, 수시는 아예 생각하지말고 정시만을 대비해야 하며, 평준화 일반고의 경우엔 
 이 악물고 내신을 3등급 혹은 4등급 초반으로 걸치게 하여 이공계 같은 경우는 지방의 아웃풋 괜찮은 공대를 노리는 것이 좋고 인문계의 경우, 경기권 하위까지라도 넣는 것이 좋습니다.
 \footnotetext{전문대를 쉽게 생각하는 경향이 있는데, 실업계의 공부하는 학생들은 내신점수받기가 훨씬 수월하므로 더 힘들 수 있습니다.}
\end{itemize}


\subsection{고등학교 2학년}

\hspace{0.3cm} 고등학교 2학년의 경우는 1학년과는 비슷하지만 현재시점에서 기회가 2번 (2학년 2학기, 3학년 1학기) 정도 밖에 남지 않았다는 것이 다릅니다. 따라서 선택의 폭이 매우 좁아집니다.

\begin{itemize}
 \item \textbf{최상위권}: 계속 내신을 유지하고 슬슬 수능기출이나 EBS를 풀면서 수능 최저등급을 맞출 생각을 해야합니다.
 \item \textbf{상위권}: 모의고사 등급에 1등급이 2개 이상 보인다, 혹은 평균 2등급 초반은 나온다면 정시로 가십시오. 그렇지 않은 경우는 남은 2번의 시험에 모든 것을 쏟아 내신을 끌어올리십시오.
 \item \textbf{중상위권}: 모의고사 등급에 2등급이 2개 이상 있다, 혹은 평균 2등급 중반까지 나온다면 정시가 유리합니다. 그렇지 않은 경우는 내신을 상위권수준으로 끌어올려야 합니다.
 \item \textbf{중위권}: 특수한 경우가 아닌 이상, 아마 더 이상 내신은 올라갈 수 없을 것입니다. 모의고사 등급이 평균 3등급 이내가 아닌 이상에야 정시도 힘듭니다. 따라서 남은 내신시험을 어떻게든 잘치고
 겨울방학부터는 빠르게 적성고사를 공략하는 것이 좋습니다.
 \item \textbf{하위권}: 내신으로는 이미 가능한 대학이 없습니다. 모의고사가 3점 중반대이내라면 정시로 가야합니다. 그렇지 않다면 내신을 어떻게든 4점대 이내로 올리고 적성고사를 대비해야 합니다.
 전문대 중에서 취업률이 괜찮은 대학을 찾아 그에 따른 대비를 하는 것도 좋습니다. (요즘은 지방 인문대보다는 전문대에서 기술을 하나 갖는 것이 더 취업에 유리합니다.)
\end{itemize}

여기까지가 대략적인 전형 결정 방법입니다. 이는 그 동안의 입시 경험들을 토대로 만들어진 것이나 당연히 완벽하지 않습니다. 따라서 참고는 하되, 너무 맹신은 하지 않으시길 바랍니다.
어디까지나 입시는 자신이 준비하는 것이므로 보다 많은 정보를 얻고 그것에 근거하여 자신이 확신하는 전형으로 가면 됩니다. 또한, 전형 하나만 보고 가는 것은 위험부담이 크므로 여러가지를
대비해두는 것이 당연히 좋습니다.

\pagebreak

\section{과목별 공부방법}

\hspace{0.3cm} 이제 전형을 정했으니 본격적으로 공부계획을 세워봅시다. 당연하게도 과목별로 공부방법은 차이가 납니다. 또한 과목안에서도 수능과 내신을 대비하는 방법은 다릅니다.
따라서 그에 따라 하나씩 정리해보도록 합시다.

\subsection{국어}

\hspace{0.3cm} 국어는 사실 매우 막막한 과목입니다. 공부를 해야하는데 어떻게 공부를 해야할 지 모르는 경우가 많습니다. 필자도 많은 시행착오를 겪었고, 그에 따른 경험을 토대로 
정리해보겠습니다.

\subsubsection{내신} 

국어과목에서 내신 점수를 잘 받으려면 필기를 열심히 하는 방법밖에 없습니다. 수업시간에 열심히 듣고 필기하고 시험기간에 다시 보는 방법뿐이죠.
따로 왕도는 존재하지 않습니다.

\subsubsection{수능}

\hspace{0.3cm} 사실 더 막막한 것은 정시대비인데, 보통의 경우는 가만히 있다가 고3 돼서야 EBS만 풀고 수능을 보는 경우가 많습니다. 그리고 그럴 경우 국어에서 발목 잡히는 경우가 상당히 많죠.
사실 국어라는 과목이 막막한 이유는 '감'이라는 것과 자신이 무엇을 못하는 지 파악을 못한 것이 큽니다. 영어나 수학의 경우에도 '감'은 항상 중요합니다만, 특히 국어에는 더욱 중요합니다. 따라서 이를 갈고 닦기 위해서는 평상시에
모의고사 기출문제를 1주일에 2회정도 풀어서 익숙해지는 방법밖에 없습니다. 또한 그렇게 풀면 당연히 자신이 약한 부분이 보이게 됩니다. 어법, 고전시가, 현대시, 현대소설 등 분명 약점이 발견될 것이고
이에 따라 얇은 개념서 한 두 권으로 보완을 해준다면 등급이 빠르게 올라가는 것을 목도할 수 있습니다. (어떤 책을 쓰면 좋은지는 뒤에서 설명하겠습니다.)

\subsubsection{요약}
\vspace{0.1cm}
\begin{itemize}
 \item \textbf{내신}: 필기를 열심히 하라. \vspace{0.1cm}
 \item \textbf{수능}: 1주일에 2회 모의고사 풀이, 약점이 발견되면 얇은 개념서로 보완.
\end{itemize}

\vspace{0.2cm}


\subsection{수학}

\hspace{0.3cm} 수학은 원래도 중요했었지만 영어가 절대평가로 전환된 지금은 혼자 변별력을 담당한다고 할 수 있을 정도로 입시에서 매우 중요한 과목이 되었습니다.
흔히 수학은 자신과 안맞는다며 포기하는 사람들이 많은데, 적어도 고등학교 수학에 국한해서는 적성에 안맞아도 성적은 아주 쉽게 올릴 수 있습니다. 바로 많이 푸는 것이죠.

\subsubsection{내신}

\hspace{0.3cm} 내신은 대비하기 아주 간단합니다. 학교마다 조금씩 차이는 있을 수 있겠으나 공통적으로 매번 내는 유형을 그대로 냅니다. 따라서 단원별로 유형을 모두 습득하고 있다면,
고득점이 안 나올 이유가 없습니다. 다만, 수학은 문제를 풀 때 풀이가 머릿속에 바로 떠오르냐는 일종의 문제풀이 '감'이 중요한데 이를 위해서는 하루도 빠짐없이 (진짜로 1년 365일 내내) 꾸준히 문제를 풀어야 합니다. 
고등학교 1학년의 경우는 하루에 50문제, 2학년은 40문제, 3학년은 30문제 정도를 매일 꾸준히 푼다면 분명히 등급상승은 따놓은 당상입니다.

\pagebreak

\subsubsection{수능}

\hspace{0.3cm} 수학의 어려움은 수능에서야 비로소 드러납니다. 분명 개념공부도 착실히 하고 문제도 열심히 풀었으나 모의고사만 보면 등급이 암울합니다. 이때부터는 단순히 문제은행식 문제만 많이
푼다고 등급이 올라가지 않습니다. 따라서 이번엔 \textbf{모의고사 등급별}로 어떻게 공부를 해야하는지 살펴보도록 합시다.

\begin{itemize}
 \item \textbf{하위권 (4등급 이하)}: 이 학생들은 아직 기초가 잡히지 않은 학생입니다. 따라서 위의 내신부분에서 말했던대로 문제은행식 문제집을 사서 하루에 30문제 이상 씩 꾸준히 푸는 연습부터
 해야 합니다. (물론 수능에 들어가는 모든 범위를 그렇게 해야합니다.) 1500문제 이상 씩 있는 문제집을 3권 정도 끝내고 나서야 이제 수능대비를 시작할 수 있습니다. \vspace{0.1cm}
 \item \textbf{중위권 (3등급)}: 3등급이 나온다는 것은 그 과목을 대충은 훑어 봤다는 것을 의미합니다. 이제부터는 단순 문제은행식 문제풀이가 아니라 수준 있는 문제들을 풀어야 합니다.
 이제 내신대비 문제집에서 벗어나 수능기출문제집을 알아봐야 합니다. 그리고 이때부터 \textbf{오답노트}라는 것을 시작하면 됩니다. 매일매일 모의고사 기출 30문제 가량을 풀고
 그날그날 오답노트에 틀린 문제와 그 풀이를 정리하는 것입니다. 이때 주의할 것은 답지를 그대로 베끼지 말고 이해한 후 답지를 보지 않고 풀이를 기록해야 한다는 것입니다. \vspace{0.1cm}
 \item \textbf{중상위권 (2등급)}: 2등급은 수능식 문제풀이에 익숙해졌다는 것을 의미합니다. 그러나 배드민턴을 치는 것은 쉬우나 배드민턴 선수가 되기는 어려운 것 처럼, 1등급이 되는 것은 다른 얘기 입니다.
 1등급이 되려면 이제 고난도 문제에 도전해야 합니다. 사관학교, 경찰대학 기출문제나 사설 모의고사, EBS N제 등 적합한 문제들은 많습니다. (이는 뒤에서 더 설명하겠습니다.) 
 아직 오답노트를 하지 않았다면 바로 시작하는 것을 추천합니다. \vspace{0.1cm}
 \item \textbf{상위권 (1등급)}: 1등급이 나왔다는 것은 나름의 공부방법을 발견했다는 것입니다. 따라서 하던대로 하되, 완벽을 지향하기 위하여 오답노트를 꾸준히 작성해야 합니다. 이는 수능 때 많은 도움이 될 것입니다.
 (실제로 항상 1등급이 나오던 학생들이 수능에서 많이 미끄러집니다.)
\end{itemize}

\subsubsection{요약}

\vspace{0.1cm}

\begin{itemize}
 \item \textbf{내신}: 하루에 30문제 이상 씩을 꾸준히 풀어라.\vspace{0.1cm}
 \item \textbf{수능}: 등급별로 푸는 문제를 달리하되, 오답노트를 작성하라.
\end{itemize}

\subsection{영어}

\hspace{0.3cm} 필자가 영어를 그리 효율적으로 공부하지는 못하였으므로 영어 공부법은 영어 선생님께 물어보는 것이 좋습니다.

\subsection{탐구}

\hspace{0.3cm} 탐구는 성적올리기가 가장 쉬운 과목입니다. 개념서를 한 번 완독한 뒤, 문제들을 열심히 풀면 등급이 올라가는 것을 단기간에도 목격할 수 있습니다.
탐구는 내신이나 수능이나 대비하는 방법이 비슷하기에 학년 별로 설명하겠습니다.

\begin{itemize}
 \item \textbf{고1}: 고등학교 1학년 탐구과목에 집착하지 말고 직접적으로 수능에 들어가는 과목을 준비해야 합니다. 자신이 수능 때 무슨 과목을 응시할 지 미리 정하여 그 중 하나의 개념서를 바로 구매해야 합니다.
 그리고 웬만하면 1 $\sim$ 2달 내로 다 읽어야 합니다. 그리고 2학기가 끝나기전에 문제집을 구매하여 1권을 다 푸는 것이 좋습니다. 적합한 교재 목록은 뒤에 소개하겠습니다.\vspace{0.1cm}
 \item \textbf{고2}: 아직 개념서를 공부하지 않았다면 이 글을 보는 즉시 바로 탐구 2개 모두 구매하여 1달 내로 다 읽으십시오. 그리고 역시나 각각 기출문제집을 사서 2학기가 끝나기 전에 다 풀어봐야 합니다.
\end{itemize}

\pagebreak

\section{공부 계획 설정하기}

\hspace{0.3cm} 이제 전형도 대략적으로 결정했고, 과목별로 공부하는 방법도 간략히 소개했으니 큰 그림을 한 번 그려봅시다. 공부계획은 각자가 설정하는 것이므로 어떠한 방법을 따로 가르쳐주지는 않겠습니다.
대신, 필자가 계획 만들 때 썼던 Guideline 을 소개하겠습니다.

\begin{enumerate}
 \item \textbf{수학과 영어는 반드시 매일 하라}: 수학 30문제, 오답노트 / 영어 듣기 30분, 단어 30개 이상은 꼭 시행해야 합니다.
 \item \textbf{국어와 탐구는 이틀에 한번 돌아가면서 하라}: 하루는 국어 기출문제 풀기, 하루는 과학 개념 공부 혹은 문제풀이.
 \item \textbf{영어 문법과 독해도 돌아가면서 하라}: 하루는 영어 문법 공부, 하루는 영어 독해 문제 풀이.
\end{enumerate}

이렇게 하면 대략 하루에 수학 30문제 + 영어 단어 30개 + 영어 듣기 30분 + (국어 or 탐구) + (문법 or 독해) 로 5개의 Contents를 공부할 수 있습니다.
대신 수학 푸는데 2시간, 영어 단어에 1시간, 듣기 30분, 국어나 탐구 공부하는데 1시간, 문법이나 독해하는데 1시간으로 5시간 반을 공부해야 하죠. 당연히 필자도 매일 5시간 반을 채우지는 못했습니다.
졸린 날도 있고 아픈 날도 있으니까요. 그래도 4시간 이상 씩 꼬박꼬박하면 처음에 비해 실력이 확연히 차이나게 됩니다. (이건 고1,2의 얘기고 고3은 하루에 최소 6시간 이상은 필수입니다.)

이는 어디까지나 참고용이며, 공부계획은 반드시 자신이 설정해야 합니다. 문,이과에 따라 방식이 달라질 것이고 자신의 상황에 따라 달라지기도 할 것입니다.
따라서 공부시간이 항상 4시간 이상이 되게끔 각자 계획을 설정해보시기 바랍니다. 
\vspace{0.2cm}

\textbf{Tip}: 시간에만 집착하지마시고 자신이 공부하려는 책들의 분량을 목표시간으로 나눠 하루 할당량을 정해야 합니다. 예를 들어 마플 문제집 1900문제를 2달에 끝낼 것이다 라고 한다면 하루에 적어도
35문제는 풀어야겠죠.

\section{입시를 위한 팁}

\vspace{0.1cm}

\begin{enumerate}
 \item \textbf{주변을 보지마라}: 주변은 자신의 경쟁자가 아닙니다. 자신이 전교 20등 안이라고 입시를 성공할 것이라고 생각하지 마십시오. 우리나라엔 공부 잘 하는 학생이 많습니다.\vspace{0.1cm}
 \item \textbf{자신을 객관적으로 파악하라}: 세상의 모든 자신은 특별합니다. 따라서 나는 할 수 있다 라고 막연히 생각하지 말고 자신의 내신등급과 모의고사가 전국 몇 \%인지 확인하십시오. \vspace{0.1cm}
 \item \textbf{학교를 믿지마라}: 학교에서 선생님들은 학생들을 안정적으로 대학에 보내기 위해 일부러 낮은 등급의 대학을 불러줍니다. 선생님의 말을 그대로 따라가다가는 언젠가 후회하게 됩니다.
 입시는 본인이 하는 겁니다. \vspace{0.1cm}
 \item \textbf{아무 정보나 믿지마라}: 인터넷에 떠도는 정보는 90\% 정도가 의미없는 정보입니다. 입시에 대하여 올바른 정보를 얻으려면 각 대학교 입학처에 나온 전년도 입시결과를 보는 것이 좋습니다.
 물론, 수만휘나 오르비 같은 대형 입시 커뮤니티나 메가스터디, 이투스 등의 대형 입시학원에서 정보를 신중히 습득하는 것도 괜찮은 방법이긴 합니다. \vspace{0.1cm}
\end{enumerate}

\pagebreak

\section*{Book Recommendation}

\hspace{0.3cm} 공부를 할 때 개념서나 문제집 선정은 상당히 큰 부분을 차지합니다. 자신의 수준에 맞지 않는 개념서나 문제집을 사용하면 오히려 공부에 거부감이 생기기까지 하죠.
그럼 어떻게 수준에 맞는 교재인지 알 수 있을까요? 정답은 자신이 서점에 가서 책을 한 번 씩 펴본 뒤 맞는 것 같은 책을 고릅니다. 물론 그렇다고 꼭 맞지는 않을 수 있습니다. 그러나 이런 경험을
계속 반복하다보면 언젠가부터 책을 똑바로 고를 수 있게 됩니다. 일단은 약간이나마 도움이 되지 않을까 해서 필자가 고등학교때 사용한 문제집 178권과 강사를 하며 괜찮아 보이는 책들을 선정해보았습니다.

\subsection*{국어}

\vspace{0.1cm}

\begin{itemize}
 \item \textbf{분야별 기출문제집}: 미래로 기출문제집 (문학, 독서), 씨리얼 (문법+화법+작문, 문학, 독서)
 \item \textbf{작품정리}: 모든 것 시리즈 (고전시가의 모든것, 현대시의 모든 것 등등) (꿈을 담는 틀)
 \item \textbf{분야별 이론}: 한끝 (화법 작문 문법편, 독서편)
 \item \textbf{공부법}: GRIT 김상훈 고등 고급 국어
\end{itemize}

\subsection*{수학}

\vspace{0.1cm}

\begin{itemize}
 \item \textbf{개념서}: 숨마쿰라우데, 셀파, 풍산자
 \item \textbf{내신문제집}: 마플시너지, 메시지, 일품
 \item \textbf{분야별 기출문제집}: 마플총정리, 씨리얼
 \item \textbf{고난도 문제집}: 특작 (가형, 나형), 하이퍼수학 최고난도 (가형, 나형), EBS N제
 \item \textbf{사설 모의고사}: 이해원 모의고사 외 오르비북스의 모의고사 시리즈
\end{itemize}

\subsection*{영어}

\vspace{0.1cm}

\begin{itemize}
 \item \textbf{단어장}: 숨마쿰라우데 WORD MANUAL, Word Master
 \item \textbf{듣기}: 자이스토리, 수능특강
 \item \textbf{문제집}: 수능다큐(어법, 독해)
 \item \textbf{EBS}: 수능특강, 인터넷수능, N제, 수능완성
\end{itemize}

\subsection*{탐구}

\vspace{0.1cm}

\begin{itemize}
 \item \textbf{개념서}: 숨마쿰라우데 (기본서), 누드교과서 (쉬운기본서), 뉴탐스런 (벼락치기용)
 \item \textbf{문제집}: 1등급만들기, 수능다큐, 자이스토리(수능기출), 씨리얼(수능기출), 수능특강
 \item \textbf{사설 모의고사}: 이투스 반전모의고사
\end{itemize}

\subsection*{연도별 기출문제집}

\vspace{0.1cm}

\begin{itemize}
 \item 입시플라이 (국어, 수학, 영어, 탐구)
\end{itemize}

%% \ackrule

%\bibliographystyle{IEEEtran}
%\bibliography{thesis}

%\section*{Biographies}

%\textbf{P. W. Wachulak} received the degree${\ldots}$ \\[6pt]
%\textbf{M. C. Marconi} received the degree${\ldots}$ \\[6pt]
%\textbf{R. A. Bartels} received the degree${\ldots}$ \\[6pt]
%\textbf{C. S. Menoni} received the degree${\ldots}$ \\[6pt]
%\textbf{J. J. Rocca} received the degree${\ldots}$


\end{document}
