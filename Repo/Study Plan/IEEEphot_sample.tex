\documentclass[final]{IEEEphot}

\usepackage{kotex}
\usepackage{setspace}
\renewcommand{\baselinestretch}{1.2} 
\jvol{xx}
\jnum{xx}
\jmonth{August}
\pubyear{2016}



\newtheorem{theorem}{Theorem}
\newtheorem{lemma}{Lemma}

\begin{document}

\title{Curriculum for 2nd Semester \\
- Mathematics -}

\author{Tae-Geun, Kim}

\affil{\textbf{B.S.}: Dept of Astronomy, Yonsei University \\
\textbf{M.S.}: Dept of Physics, Yonsei University}

\maketitle


\begin{abstract}
 어느 새 2학기가 되었습니다. 학기 초에 목표를 바르게 잡아야 학기 말까지 열심히 공부할 수 있습니다. 목표를 잡고 계획을 세우는데는 본인 뿐만 아니라 학원의 계획도 영향을 미치기에
 정진학원의 2학기 계획과 커리큘럼에 대하여 알려드리고자 합니다. 
\end{abstract}


\section{고등학교 1학년}

\hspace{0.3cm} 고등학교 1학년의 2학기는 상당히 중요합니다. 1학기야 적응하는 시기라고 쳐도, 2학기 부터는 한 번의 실수가 고등학교 3년을 결정할 수도 있습니다. 반대로, 1학년 2학기를
성공적으로 보낸다면 앞으로 입시준비하는데 매우 큰 도움이 될 수 있습니다. 

그럼 이제 본격적으로 수학 커리큘럼에 대하여 얘기해보겠습니다. 중학교와 달리 고등학교 수학은 단순히 문제만 많이 풀고 선행만 빠르게 해서는 해결되지 않습니다. 수학 자체가 복잡해짐에 따라
내신과 수능에서 요구하는 수학능력에 괴리가 있으며, 과목과 과목들이 연계되기에 앞의 것이 완벽히 되지 않으면 선행학습을 해도 수박 겉핥기에 지나지 않기 때문입니다. 따라서 제일 먼저 필요한 것은
개념을 완벽히 이해하는 것과 그것의 내신에서의 활용과 수능에서의 활용을 모두 잡는 것입니다. 그 이후에 조금은 발빠른 선행을 병행한다면 학습효과는 상당할 것입니다. 
따라서 정진학원 고등학교 1학년 수학의 2학기 계획은 다음과 같습니다.

\begin{enumerate}
 \item \textbf{개념}: 8월 25일 부로 중간고사 시험범위(집합, 명제, 함수, 유리,무리함수)가 끝이 납니다. 학생들의 기억력의 한계로 기말고사 부분은 중간 직후에 다룰 예정입니다. \vspace{0.1cm}
 \item \textbf{문제}: 9월 한 달간 마플 시너지(1900문항)를 이용하여 하루에 40문제씩 꼬박꼬박 문제를 풀게 할 것입니다. 힘들겠지만 이것이 원래 기본입니다.\vspace{0.1cm}
 \item \textbf{중간대비}: 매일 40문제씩을 풀면 당연히 시험은 잘 볼 수밖에 없습니다. 그러나 항상 방심은 금물이므로 교과서 풀이와 모의 시험을 볼 것입니다.
 \item \textbf{수능대비}: 9월에 매일 40문제씩 푸는 것 뿐 아니라 씨리얼(수능기출문제집)을 이용하여 매주 수능식 문제풀이에 익숙하게 할 것입니다. 더불어 11월 모의고사의 
 시험범위는 중간고사보다 한단원 많은 수열까지이므로 중간고사가 끝난 후 바로 빠르게 진도를 나가고 본격적으로 11월 모의고사에 대비할 것입니다. \vspace{0.1cm}
 \item \textbf{기말대비}: 기말고사는 12월 중순 혹은 말에 보는 것이 관례입니다. 따라서 11월 모의고사가 끝난 후 중간에서 했던 것과 같은 방식으로 모든 유형을 익혀 고득점을 노릴 것입니다. \vspace{0.1cm}
 \item \textbf{선행}: 고1때 배우는 수학 1,2는 고2부터는 다시 배우지 않습니다. 그러나 수능에는 항상 사용되는 개념이므로 완벽히 해두지 않으면 다른 과목을 아무리 잘 해도 수능에서는 도태됩니다.
 따라서 내신과 수능을 모두 다진 기말고사가 끝나는 시점부터 미적분1 선행학습을 하여도 늦지 않습니다. (그렇게 하더라도 2학년 올라가기 전까지 한 번은 끝낼 수 있습니다.)
\end{enumerate}

그리하여 기대되는 성과는 다음과 같습니다.

\begin{enumerate}
 \item \textbf{내신등급의 가파른 상승}: 만약 학생들이 철저히 40문제 푸는 것을 따라온다면 분명히 1등급이 나올 것입니다. 만약 못하더라도 적어도 1$\sim$2등급의 상승을 기대할 수 있습니다. \vspace{0.1cm}
 \item \textbf{본격적인 수능 대비}: 모의고사 등급은 내신 등급과는 달리 한 번에 잘 오르지 않지만, (물론 내신도 바로 오르긴 힘듭니다만) 그 전까지 학생들이 대비하지 않았던 모의고사를 대비하면서 
 정시로의 가능성을 보고 창의적인 사고력을 키우는 것이 목표입니다. (만약 11월 모의고사에서 수학이 3등급 이상이 나온다면 가능성이 보이는 겁니다.) \vspace{0.1cm}
 \item \textbf{공부 방법 제시}: 숙제가 있으면 하고, 없으면 하지 않았던 공부방식을 매일 풀어야 하는 과제를 부여함으로 기본적인 자습 태도를 확립할 수 있습니다.
\end{enumerate}

\section{고등학교 2학년}

\hspace{0.3cm} 고2는 현재 중간고사 범위까지는 진도를 끝냈기에 8월 내로 내신 문제풀이를 마무리 하고 수능 대비체제로 전향하려 합니다. 따라서 이번에 마플 총정리(수능기출 1300문항)를 9월 한 달간
매일 30문제씩 꾸준히 풀고 질의응답하면서 수능에 대한 감을 늘리고자 합니다.

\begin{enumerate}
 \item \textbf{개념}: 미적분은 중간고사 범위는 이미 끝났고 9월 내로 기말고사 범위까지 모두 끝낼 것입니다. 더불어 기하와 벡터 역시 꾸준히 나갈 예정입니다. \vspace{0.1cm}
 \item \textbf{문제}: 9월 한 달간 수능 문제로 이루어진 마플 총정리 미적분1을 풀 것입니다. 이것으로 11월 모의고사 대비하는 데에 도움이 될 것입니다. \vspace{0.1cm}
 \item \textbf{중간대비}: 중간고사는 교과서에서 많이 나오는 솔터고등학교 특성 상 교과서 문제 유형들을 분석하여 이번에도 1등급을 목표로 할 것입니다. \vspace{0.1cm}
 \item \textbf{수능대비\&선행}: 11월 모의고사는 미적분1 전체와 미적분2 반 정도 입니다. 그러나 학교에서는 미적분1도 모두 끝내지 않으므로 중간고사 직후 학원 단독으로 미적분2 특강을 하여 대비할 것입니다. 
 이후, 기말고사가 끝나는 시점부터 확률과통계와 미적분2를 병행하여 고3 올라가기 전에 완벽히 선행을 마칠 것입니다.\vspace{0.1cm}
 \item \textbf{기말대비}: 기말범위도 매우 빠르게 끝나므로 수능문제 위주로 공부하다가 시험기간에 교과서를 완벽히 풀고 추가적으로 모의시험을 보는 것으로 대비할 것입니다.
\end{enumerate}

고2의 이번 목표는 다음과 같습니다.

\begin{enumerate}
 \item \textbf{미적분1, 기하와벡터 1등급}: 저번학기에는 비록 기하와 벡터만 100점을 맞았지만 이번엔 둘 모두 잡을 수 있게 할 것입니다. \vspace{0.1cm}
 \item \textbf{11월 모의고사 수학 3등급이내}: 수능 문제로 미적분1을 훈련하고 학교 내의 다른 학생들은 하지 않는 미적분2를 선행하면서 전교에서는 적어도 3등 이내, 등급은 3등급이 목표입니다. \vspace{0.1cm}
 \item \textbf{자습 태도 개선}: 역시나 불규칙적으로 공부했던 방식에서 탈피하여 매일 매일 계획적으로 자습하는 태도를 확림하는 것이 목표입니다.
\end{enumerate}

\end{document}
